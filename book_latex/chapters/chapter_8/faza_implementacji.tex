\section{Faza Implementacji}
Implementacja projektu odbyła się w sprintach, które trwało od 2 do 4 tygodni. Członkowie zespołu w ramach każdego sprintu mieli do wykonania zadania, które były przypisane do nich w narzędziu do zarządzania projektem jira.

\subsection{Sprint 1}

\subsubsection{Opis Sprintu}
Pierwszy przyrost dotyczył podstawowej konfiguracji projektu z wykorzystaniem frameworka FastAPI. Na początku został utworzony projekt wraz z podstawową strukturą katalogów , następnie powstał plik readme, który zawierał instrukcję uruchomienia projektu. Kolejnym wykonanym krokiem było utworzenie modeli karty fiszek i talii. Ostatnim zadaniem wykonanym w przyroście było opakowanie projektu w kontener dockerowy.

\subsubsection{Wykonane zadania}

\begin{longtable}{|p{0.6\linewidth}|p{0.3\linewidth}|}
\hline
\textbf{Zadanie} & \textbf{Wykonawca} \\
\hline
\endfirsthead
\hline
\textbf{Zadanie} & \textbf{Wykonawca} \\
\hline
\endhead
\hline
\caption{Zadania wykonane w sprincie 1}
\endfoot
\hline
Utworzenie kontenera backendu & Jakub \\
\hline
Aktualizacja Readme & Oliwier \\
\hline
[BACKEND] Dodanie 'connector' dla łączności z bazą danych & Jakub \\
\hline
[BACKEND] Utworzenie modelu autoryzacji i użytkownika & Jakub \\
\hline
[BACKEND] Utworzenie cli i funkcji do tworzenia modeli w bazie & Jakub \\
\hline
[BACKEND] Dodanie app\_middleware'y & Jakub \\
\hline
[BACKEND] Utworzenie modelu tali oraz karty & Oliwier \\
\hline
[BACKEND] Naprawienie ścieżki importów projektu & Oliwier
\hline

\end{longtable}

\subsection{Sprint 2}

\subsubsection{Opis Sprintu}
Drugi przyrost był skupiony na utworzeniu widoku logowania i rejestracji dla aplikacji mobilnej, a także na implementacji tokenu autoryzacji i przypisaniu ról użytkownikom systemu.

\subsubsection{Wykonane zadania}

\begin{longtable}{|p{0.6\linewidth}|p{0.3\linewidth}|}
\hline
\textbf{Zadanie} & \textbf{Wykonawca} \\
\hline
\endfirsthead

\hline
\textbf{Zadanie} & \textbf{Wykonawca} \\
\hline
\endhead

\hline
\endfoot

\hline
\caption{Zadania wykonane w sprincie 2}
\endlastfoot

[BACKEND] Poprawienie middleware dla jwt & Oliwier \\
\hline
[MOBILE] utworzyć style scss i zaimportować & Daniel \\
\hline
[MOBILE] Wstepny widok logowanie i rejestracji bez funcjonalnosci (mobilka) & Daniel \\
\hline
[MOBILE] Dodać serwis logowania i rejestracji & Jakub \\
\hline
[BACKEND] Utworzyć wstępne fixtury & Jakub \\
\hline
[BACKEND] Utworzyć endpoint do resetowania hasła & Jakub \\
\hline
[MOBILE] Utworzyć wstępne pliki i konfiguracje dla aplikacji mobilnej & Jakub \\
\hline
[BACKEND] Utworzenie loggera & Jakub \\
\hline
[BACKEND] Poprawienie ścieżki dla api & Jakub \\
\hline
[BACKEND] Dodanie nowej dependencji dla roli & Jakub \\
\hline
[BACKEND] Dodanie tokenu na czarną listę & Jakub \\
\hline

\end{longtable}

\subsection{Sprint 3}

\subsubsection{Opis Sprintu}
Trzeci przyrost był skupiony na zadaniach związanych z uporządkowaniem kodu aplikacji mobilną. Ważnym krokiem dotyczącem strony webowej było utworzenie strony domowej. Backend został rozbudowany o nowe endpointy związane z talią, został utworzony role checker do sprawdzania roli użytkownika aplikacji.

\subsubsection{Wykonane zadania}

\begin{longtable}{|p{0.6\linewidth}|p{0.3\linewidth}|}
\hline
\textbf{Zadanie} & \textbf{Wykonawca} \\
\hline
\endfirsthead

\hline
\textbf{Zadanie} & \textbf{Wykonawca} \\
\hline
\endhead

\hline
\endfoot

\hline
\caption{Zadania wykonane w sprincie 3}
\endlastfoot

[MOBILE] Dodać możliwość rejestracji & Jakub \\
\hline
[MOBILE] Dodać panel użytkownika & Jakub \\
\hline
[MOBILE] Zrobić porządek w kodzie & Jakub \\
\hline
[MOBILE] Dodać walidacje hasła & Jakub \\
\hline
[MOBILE] Zaktualizować serwisy z nową metodą request & Jakub \\
\hline
[WEB] Okno logowania & Wiktor \\
\hline
[BACKEND] Poprawić endpoint decs & Oliwier \\
\hline
[WEB] Okno rejestracji & Wiktor \\
\hline
[MOBILE] Dodać interface dla zwracanych danych w metodzie request & Jakub \\
\hline
[BACKEND] napisanie endpointów dla fiszek & Oliwier \\
\hline
[WEB] Utworzenie kontenera docker dla NodeJS & Oliwier \\
\hline
[BACKEND] Poprawić dependencje dla RoleCheckera & Jakub \\
\hline
[MOBILE] Utworzenie metody request & Jakub \\
\hline
[MOBILE] Utworzyć stronę domową po zalogowaniu & Daniel \\
\hline
[WEB] Utworzyć stronę domową po zalogowaniu & Oliwier \\
\hline
[MOBILE] Przerzucić regexy do katalogu validator & Daniel \\
\hline
[MOBILE] Poprawić widok logowania i rejestracji & Daniel \\
\hline
[WEB] Utworzenie strony do tworzenia decku & Oliwier \\
\hline

\end{longtable}

\subsection{Sprint 4}

\subsubsection{Opis Sprintu}
W czwartym przyroście aplikacja webowa została w dużym stopniu rozbudowana o nowe widoki, a także została połączona z warstwą backend, umożliwiło to komunikację strony internetowej z bazą danych w celu pobierania i tworzenia danych potrzebnych do rejestracji i logowania użytkownika. Rozbudowa aplikacji mobilnej  była skupiona na profilu użytkownika, zostały dodane funkcjonalności związane ze zmianą i aktualizacją danych.

\subsubsection{Wykonane zadania}

\begin{longtable}{|p{0.6\linewidth}|p{0.3\linewidth}|}
\hline
\textbf{Zadanie} & \textbf{Wykonawca} \\
\hline
\endfirsthead

\hline
\textbf{Zadanie} & \textbf{Wykonawca} \\
\hline
\endhead

\hline
\endfoot

\hline
\caption{Zadania wykonane w sprincie 4}
\endlastfoot

[MOBILE] Utworzyć loader & Jakub \\
\hline
[MOBILE] Dodać modal aby potwierdzić hasłem & Jakub \\
\hline
[BACKEND] Poprawa modeli talii i fiszki & Oliwier \\
\hline
[MOBILE] Dodać walidacje hasła & Jakub \\
\hline
[MOBILE] Widok dla zmiany email & Jakub \\
\hline
[MOBILE] Widok dla zmiany hasła & Jakub \\
\hline
[MOBILE] Widok dla zmiany nazwy użytkownika & Jakub \\
\hline
[BACKEND] Dodać endpointy na zaktualizowanie danych użytkownika & Jakub \\
\hline
[MOBILE] Widok tworzenia decku & Daniel \\
\hline
[MOBILE] Bottom tab navigator & Daniel \\
\hline
[MOBILE] Widok my decks & Daniel \\
\hline
[WEB] Połączenie strony tworzenia fiszek z backendem & Oliwier \\
\hline
[WEB] Połączenie strony domowej z backendem & Oliwier \\
\hline
[WEB] Utworzenie strony my decks & Oliwier \\
\hline
[WEB] Wylogowanie użytkownika & Oliwier \\
\hline
[WEB] Utworzenie customowego okna dla alertów & Oliwier \\
\hline
[WEB] Utworzenie widoku strony do nauki z talii fiszek & Oliwier \\
\hline

\end{longtable}

\subsection{Sprint 5}

\subsubsection{Opis Sprintu}

Do aplikacji został dodany endpoint, umożliwiający komunikację z czatem GPT. Pozwoliło to na dodanie do strony webowej funkcjonalności związanej z generowaniem treści. Zostały także zaimplementowany tryb uczenia się, który pozwala na dzielenie fiszek na zapamiętane i nie zapamiętane. Do aplikacji mobilnej zostało dodane usuwanie konta użytkownika oraz poprawiona została struktura kodu.


\subsubsection{Wykonane zadania}

\begin{longtable}{|p{0.6\linewidth}|p{0.3\linewidth}|}
\hline
\textbf{Zadanie} & \textbf{Wykonawca} \\
\hline
\endfirsthead

\hline
\endfoot

\hline
\caption{Zadania wykonane w sprincie 5}
\endlastfoot

[MOBILE] Dodać usuwanie konta & Jakub \\
\hline
[BACKEND] Dodał podstawowe fixtury dla decków & Jakub \\
\hline
[WEB] Zmiana hasła użytkownika & Wiktor \\
\hline
[WEB] Utworzenie strony do sterowania głosem talia fiszek & Oliwier \\
\hline
[BACKEND] Dodanie endpointu do obsługi czatu GPT & Oliwier \\
\hline
[WEB] Dodać generowanie treści przy użyciu chatu GPT & Oliwier \\
\hline
[BACKEND] Dodanie do usera kolumny dla avatara, poprawa dodanie w flashcard kolumny is memorized & Oliwier \\
\hline
[WEB] Dodanie widoku dla not memorized flashcards & Oliwier \\
\hline
[BACKEND] Dodanie endpointów do flashcards, które filtrują zapamiętane i nie zapamiętane karty & Oliwier \\
\hline
[WEB] Dodanie trybu uczenia & Oliwier \\
\hline
[WEB] Dodać opcję edycji fiszki i możliwość udostępnienia decku & Oliwier \\
\hline
[MOBILE] Poprawienie nazewnictwa w kodzie nawigacji & Daniel \\
\hline
[MOBILE] Naprawa struktury ekranów & Daniel \\
\hline

\end{longtable}

\subsection{Sprint 6}

\subsubsection{Opis Sprintu}

Do aplikacji mobilnej został dodany ekran tworzenia talii fiszek, następne dodane widoki były związane z trybem uczenia się. Aplikacja webowa została rozbudowana o ranking talii i użytkowników. Na stronie webowej zostały dodane funkcjonalności związane z aktualizacją danych użytkownika. Utworzony został model nlp do rozumienia semantyki słów wykorzystany do sterowania talią fiszek przy użyciu komend głosowych. Dla backendu zostały napisany testy integracyjne uruchamiany przy użyciu biblioteki pytest.

\subsubsection{Wykonane zadania}

\begin{longtable}{|p{0.6\linewidth}|p{0.3\linewidth}|}
\hline
\textbf{Zadanie} & \textbf{Wykonawca} \\
\hline
\endfirsthead

\hline


\hline
\endfoot

\caption{Zadania wykonane w sprincie 6}
\endlastfoot

[MOBILE] Dodać możliwość update'u avatara & Jakub \\
\hline
[MOBILE] Naprawić błąd z query w widoku decklist & Jakub \\
\hline
[WEB] Profil użytkownika & Wiktor \\
\hline
[BACKEND] Dodać celery do aplikacji & Jakub \\
\hline
[WEB] Usunięcie konta użytkownika & Wiktor \\
\hline
[WEB] Zmiana email użytkownika & Wiktor \\
\hline
[WEB] Zmiana nicku użytkownika & Wiktor \\
\hline
[BACKEND] Utworzyć metodę do wysyłania emaila & Jakub \\
\hline
[BACKEND] Utworzyć templatki dla maila & Jakub \\
\hline
[MOBILE] Detale usera z rankingu & Jakub \\
\hline
[MOBILE] Spiąć "My Decks" z API & Daniel \\
\hline
[MOBILE] Spiąć "Create Decks" z API & Daniel \\
\hline
[MOBILE] Dodać ekran podglądu fiszek & Daniel \\
\hline
[MOBILE] Dodać ekran podglądu decku & Daniel \\
\hline
[WEB] Ranking użytkowników & Oliwier \\
\hline
[MOBILE] Dodać listę udostępnionych talii & Jakub \\
\hline
[MOBILE] Spiąć podgląd decku z API & Daniel \\
\hline
[MOBILE] Dodać ekran tworzenia fiszki & Daniel \\
\hline
[MOBILE] Spiąć ekran tworzenia fiszki z API & Daniel \\
\hline
[MOBILE] Dodać service flashcards & Daniel \\
\hline
[MOBILE] Utworzyć component pobierania danych z API & Daniel \\
\hline
[MOBILE] Dodać edycję i usunięcie fiszki & Daniel \\
\hline
[MOBILE] Dodać ekran settings dla decku & Daniel \\
\hline
[WEB] Poprawa rankingów & Oliwier \\
\hline
[MOBILE] Dodać i spiąć memorized flashcards & Daniel \\
\hline
[BACKEND] Testy integracyjne & Oliwier \\
\hline
[WEB] Utworzenie strony public decks & Oliwier \\
\hline
[WEB] Dodanie obsługi złożonych komend głosowych & Oliwier \\
\hline
[BACKEND] Utworzenie modelu NLP do rozumienia semantyki słów & Oliwier \\
\hline

\end{longtable}