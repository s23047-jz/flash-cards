\chapter{Implementacja}

Implementacja projektu odbyła się metodyką przyrostowo-ewolucyjną. Metodyka ta pozwala na szybką adaptację i elastyczność, która jest kluczowa w przypadku projektów, które mogą ulegać zmianom w czasie realizacji. Tworzenie systemu odbyło się w dwóch etapach, fazie planowanie i fazie implementacji. Faza planowania dotyczyła określenia celu projektu, funkcjonalności systemu i wypełnienia dokumentacji. Faza implementacji odbywała się w sprintach, które zazwyczaj trwały od 2 do 3 tygodni.

\section{Faza planowania}
Na samym początku była burza mózgów w celu wymyślenia tematu naszego projektu. Po analizie potencjalnych tematów zespół zdecydował się na aplikację do uczenia się z wykorzystaniem metody fiszek. Następnym krokiem było wypełnienie karty projektu, zespół musiał określić cele projektu, miary sukcesu i główne funkcjonalności. Po określeniu ogólnych założeń dotyczących projektu, zespół zajął się wypełnieniem dokumentu założeń wstępnych, który dotyczył opisu naszego problemu, analizy konkurencji i ogólnej wizji konstrukcyjnej. Ostatnim krokiem w fazie planowania było wypełnienie specyfikacji wymagań systemowych, która dotyczyła szczegółowego opisu wymagań dotyczących naszego projektu. Po ukończeniu planowania zespół był gotowy do przejścia w fazę implementacji.

\section{Faza implementacji}
TBC