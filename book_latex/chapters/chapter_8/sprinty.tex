\subsection{Sprint 1}

Pierwszy przyrost dotyczył podstawowej konfiguracji projektu z wykorzystaniem frameworka \\{FastAPI}. Na początku został utworzony projekt wraz z podstawową strukturą katalogów, następnie powstał plik readme, który zawierał instrukcję uruchomienia projektu. Kolejnym wykonanym krokiem było utworzenie modeli fiszek i talii. Ostatnim zadaniem wykonanym w przyroście było opakowanie projektu w kontener dockerowy.

\subsubsection{Wykonane zadania:}

\begin{table}[H]
\centering
\begin{tabularx}{\textwidth}{|p{0.8\textwidth}|X|}
    \hline
    \textbf{Zadanie} & \textbf{Wykonawca} \\
    \hline
    Utworzenie kontenera backendu & Jakub \\
    \hline
    Aktualizacja Readme & Oliwier \\
    \hline
    [BACKEND] Dodanie 'connector' dla łączności z bazą danych & Jakub \\
    \hline
    [BACKEND] Utworzenie modelu autoryzacji i użytkownika & Jakub \\
    \hline
    [BACKEND] Utworzenie cli i funkcji do tworzenia modeli w bazie & Jakub \\
    \hline
    [BACKEND] Dodanie app\_middleware'y & Jakub \\
    \hline
    [BACKEND] Utworzenie modelu tali oraz karty & Oliwier \\
    \hline
    [BACKEND] Naprawienie ścieżki importów projektu & Oliwier \\
    \hline
\end{tabularx}
        \caption{Zadania wykonane w sprincie 1.}
\end{table}

\subsection{Sprint 2}

Drugi przyrost był skupiony na utworzeniu widoku logowania i rejestracji dla aplikacji
mobilnej, a także na implementacji tokenu autoryzacji i przypisaniu ról użytkownikom systemu.

\subsubsection{Wykonane zadania:}

\begin{table}[H]
\centering
\begin{tabularx}{\textwidth}{|p{0.8\textwidth}|X|}
    \hline
    \textbf{Zadanie} & \textbf{Wykonawca} \\
    \hline
    [BACKEND] Poprawienie middleware dla jwt & Oliwier \\
    \hline
    [MOBILE] Utworzyć style scss i zaimportować & Daniel \\
    \hline
    [MOBILE] Wstępny widok logowania i rejestracji bez funkcjonalności (mobilka) & Daniel \\
    \hline
    [MOBILE] Dodać serwis logowania i rejestracji & Jakub \\
    \hline
    [BACKEND] Utworzyć wstępne fixtury & Jakub \\
    \hline
    [BACKEND] Utworzyć endpoint do resetowania hasła & Jakub \\
    \hline
    [MOBILE] Utworzyć wstępne pliki i konfiguracje dla aplikacji mobilnej & Jakub \\
    \hline
    [BACKEND] Utworzenie loggera & Jakub \\
    \hline
    [BACKEND] Poprawienie ścieżki dla api & Jakub \\
    \hline
    [BACKEND] Dodanie nowej dependencji dla roli & Jakub \\
    \hline
    [BACKEND] Dodanie tokenu na czarną listę & Jakub \\
    \hline
\end{tabularx}
    \caption{Zadania wykonane w sprincie 2.}
\end{table}

\subsection{Sprint 3}

Trzeci przyrost był skupiony na zadaniach związanych z uporządkowaniem kodu aplikacji mobilną. Ważnym krokiem dotyczącym strony webowej było utworzenie strony domowej. Backend został rozbudowany o nowe endpointy związane z talią, został utworzony role checker do sprawdzania roli użytkownika aplikacji.

\subsubsection{Wykonane zadania:}

\begin{table}[H]
\centering
\begin{tabularx}{\textwidth}{|p{0.8\textwidth}|X|}
    \hline
    \textbf{Zadanie} & \textbf{Wykonawca} \\
    \hline
    [MOBILE] Dodać możliwość rejestracji & Jakub \\
    \hline
    [MOBILE] Dodać panel użytkownika & Jakub \\
    \hline
    [MOBILE] Zrobić porządek w kodzie & Jakub \\
    \hline
    [MOBILE] Dodać walidacje hasła & Jakub \\
    \hline
    [MOBILE] Zaktualizować serwisy z nowa metoda request & Jakub \\
    \hline
    [WEB] Okno logowania & Wiktor \\
    \hline
    [BACKEND] Poprawić endpoint decs & Oliwier \\
    \hline
    [WEB] Okno rejestracji & Wiktor \\
    \hline
    [MOBILE] Dodać interface dla zwracanych danych w metodzie request & Jakub \\
    \hline
    [BACKEND] napisanie endpointów dla fiszek & Oliwier \\
    \hline
    [WEB] Utworzenie kontenera docker dla NodeJS & Oliwier \\
    \hline
    [BACKEND] Poprawić dependencje dla RoleCheckera & Jakub \\
    \hline
    [MOBILE] Utworzenie metody request & Jakub \\
    \hline
    [MOBILE] Utworzyc strone domowa po zalogowaniu & Daniel \\
    \hline
    [WEB] Utworzyc strone domowa po zalogowaniu & Oliwier \\
    \hline
    [MOBILE] Przerzucić regexy z do katalogu validator & Daniel \\
    \hline
    [MOBILE] Poprawić widok logowania i rejestracji & Daniel \\
    \hline
    [WEB] Utworzenie strony do tworzenia decku & Oliwier \\
    \hline
\end{tabularx}
        \caption{Zadania wykonane w sprincie 3.}
\end{table}

\subsection{Sprint 4}

W czwartym przyroście aplikacja webowa została w dużym stopniu rozbudowana o nowe widoki, a także została połączona z warstwą backend, umożliwiło to komunikację strony internetowej z bazą danych w celu pobierania i tworzenia danych potrzebnych do rejestracji i logowania użytkownika. Rozbudowa aplikacji mobilnej  była skupiona na profilu użytkownika, zostały dodane funkcjonalności związane ze zmianą i aktualizacją danych.

\subsubsection{Wykonane zadania:}

\begin{table}[H]
\centering
\begin{tabularx}{\textwidth}{|p{0.8\textwidth}|X|}
    \hline
    \textbf{Zadanie} & \textbf{Wykonawca} \\
    \hline
    [MOBILE] Utworzyć loader & Jakub \\
    \hline
    [MOBILE] Dodać modal aby potwierdzić hasłem & Jakub \\
    \hline
    [BACKEND] Poprawa modeli talii i fiszki & Oliwier \\
    \hline
    [MOBILE] Dodać walidacje hasła & Jakub \\
    \hline
    [MOBILE] Widok dla zmiany e-mail & Jakub \\
    \hline
    [MOBILE] Widok dla zmiany hasła & Jakub \\
    \hline
    [MOBILE] Widok dla zmiany nazwy użytkownika & Jakub \\
    \hline
    [BACKEND] Dodać endpointy na zaktualizowanie danych użytkownika & Jakub \\
    \hline
    [MOBILE] Widok tworzenia decku & Daniel \\
    \hline
    [MOBILE] Bottom tab navigator & Daniel \\
    \hline
    [MOBILE] Widok my decks & Daniel \\
    \hline
    [WEB] Połączenie strony tworzenia fiszek z backendem & Oliwier \\
    \hline
    [WEB] Połączenie strony domowej z backendem & Oliwier \\
    \hline
    [WEB] Utworzenie strony my decks & Oliwier \\
    \hline
    [WEB] Wylogowanie użytkownika & Oliwier \\
    \hline
    [WEB] Utworzenie customowego okna dla alertów & Oliwier \\
    \hline
    [WEB] Utworzenie widoku strony do nauki z talii fiszek & Oliwier \\
    \hline
\end{tabularx}
            \caption{Zadania wykonane w sprincie 4.}
\end{table}

\subsection{Sprint 5}

Do aplikacji został dodany endpoint, umożliwiający komunikację z ChatGPT. Pozwoliło to na dodanie do strony webowej funkcjonalności związanej z generowaniem treści. Zostały także zaimplementowany tryb uczenia się, który pozwala na dzielenie fiszek na zapamiętane i niezapamiętane. Do aplikacji mobilnej zostało dodane usuwanie konta użytkownika oraz poprawiona została struktura kodu.

\subsubsection{Wykonane zadania:}

\begin{table}[H]
\centering
\begin{tabularx}{\textwidth}{|p{0.8\textwidth}|X|}
    \hline
    \textbf{Zadanie} & \textbf{Wykonawca} \\
    \hline
    [MOBILE] Dodać usuwanie konta & Jakub \\
    \hline
    [BACKEND] Dodał podstawowe fixtury dla decków & Jakub \\
    \hline
    [WEB] Zmiana hasła użytkownika & Wiktor \\
    \hline
    [WEB] Utworzenie strony do sterowania głosem talia fiszek & Oliwier \\
    \hline
    [BACKEND] Dodanie endpointu do obsługi ChatGPT & Oliwier \\
    \hline
    [WEB] Dodać generowanie treści przy użyciu ChatGPT & Oliwier \\
    \hline
    [BACKEND] Dodanie do usera kolumny dla avatara poprawa dodanie w flashcard kolumny is memorized & Oliwier \\
    \hline
    [WEB] Dodanie widoku dla not memorized flashcards & Oliwier \\
    \hline
    [BACKEND] Dodanie endpointów do flashcards które filtrują zapamiętane i niezapamiętane karty & Oliwier \\
    \hline
    [WEB] Dodanie trybu uczenia & Oliwier \\
    \hline
    [WEB] Dodać opcje edycji fiszki i możliwość udostępnienia decku & Oliwier \\
    \hline
    [MOBILE] Poprawienie nazewnictwa w kodzie nawigacji & Daniel \\
    \hline
    [MOBILE] Naprawa struktury ekranów & Daniel \\
    \hline
\end{tabularx}
                \caption{Zadania wykonane w sprincie 5.}
\end{table}

\subsection{Sprint 6}

Do aplikacji mobilnej został dodany ekran tworzenia talii fiszek, następne dodane widoki były związane z trybem uczenia się. Aplikacja webowa została rozbudowana o ranking talii i użytkowników. Na stronie webowej zostały dodane funkcjonalności związane z aktualizacją danych użytkownika. Utworzony został model nlp do rozumienia semantyki słów wykorzystany do sterowania talią fiszek przy użyciu komend głosowych. Dla backendu zostały napisany testy integracyjne uruchamiany przy użyciu biblioteki "pytest".

\subsubsection{Wykonane zadania:}

\begin{table}[H]
\centering
\begin{tabularx}{\textwidth}{|p{0.8\textwidth}|X|}
    \hline
    \textbf{Zadanie} & \textbf{Wykonawca} \\
    \hline
    [MOBILE] Dodać możliwość update'u avatara & Jakub \\
    \hline
    [MOBILE] Naprawić błąd z query w widoku decklist & Jakub \\
    \hline
    [WEB] Profil użytkownika & Wiktor \\
    \hline
    [BACKEND] Dodać celery do aplikacji & Jakub \\
    \hline
    [WEB] Usunięcie konta użytkownika & Wiktor \\
    \hline
    [WEB] Zmiana e-mail użytkownika & Wiktor \\
    \hline
    [WEB] Zmiana nicku użytkownika & Wiktor \\
    \hline
    [BACKEND] Utworzyć metodę do wysyłania e-maila & Jakub \\
    \hline
    [BACKEND] Utworzyć templatki dla maila & Jakub \\
    \hline
    [MOBILE] Detale usera z rankingu & Jakub \\
    \hline
    [MOBILE] Spiąć "My Decks" z API & Daniel \\
    \hline
    [MOBILE] Spiąć "Create Decks" z API & Daniel \\
    \hline
    [MOBILE] Dodać ekran podglądu fiszek & Daniel \\
    \hline
    [MOBILE] Dodać ekran podglądu decku & Daniel \\
    \hline
    [WEB] Ranking użytkowników & Oliwier \\
    \hline
    [MOBILE] Dodać listę udostępnionych talii & Jakub \\
    \hline
    [MOBILE] Spiąć podgląd decku z API & Daniel \\
    \hline
    [MOBILE] Dodać ekran tworzenia fiszki & Daniel \\
    \hline
    [MOBILE] Spiąć ekran tworzenia fiszki z API & Daniel \\
    \hline
    [MOBILE] Dodać service flashcards & Daniel \\
    \hline
    [MOBILE] Utworzyć component pobierania danych z API & Daniel \\
    \hline
    [MOBILE] Dodać edycję i usunięcie fiszki & Daniel \\
    \hline
    [MOBILE] Dodać ekran settings dla decku & Daniel \\
    \hline
    [WEB] Poprawa rankingów & Oliwier \\
    \hline
    [MOBILE] Dodać i spiąć memorized flashcards & Daniel \\
    \hline
    [BACKEND] Testy integracyjne & Oliwier \\
    \hline
    [WEB] Utworzenie strony public decks & Oliwier \\
    \hline
    [WEB] Dodanie obsługi złożonych komend głosowych & Oliwier \\
    \hline
    [BACKEND] Utworzenie modelu nlp do rozumienia semantyki słów & Oliwier \\
    \hline
\end{tabularx}
                    \caption{Zadania wykonane w sprincie 6.}
\end{table}

\subsection{Sprint 7}

Sprint 7 był ostatnim przyrostem w naszym projekcie. Zespół był skupiony na dopracowaniu ostatecznej wersji aplikacji i poprawie występujących błędów. Na platformie Azure został zainicjowany, uruchomiony i skonfigurowany serwer wirtualny, który obsługuję stornę internetową i backend.

\subsubsection{Wykonane zadania:}

\begin{longtable}{|p{0.8\textwidth}|p{0.2\textwidth}|}
    \hline
    \textbf{Zadanie} & \textbf{Wykonawca} \\
    \hline
    [MOBILE] Dodać możliwość update'u avatara & Jakub \\
    \hline
    [MOBILE] Naprawić błąd z query w widoku decklist & Jakub \\
    \hline
    [WEB] Profil użytkownika & Wiktor \\
    \hline
    [BACKEND] Dodać celery do aplikacji & Jakub \\
    \hline
    [WEB] Usunięcie konta użytkownika & Wiktor \\
    \hline
    [WEB] Zmiana e-mail użytkownika & Wiktor \\
    \hline
    [WEB] Zmiana nicku użytkownika & Wiktor \\
    \hline
    [BACKEND] Utworzyć metodę do wysyłania e-maila & Jakub \\
    \hline
    [BACKEND] Utworzyć templatki dla maila & Jakub \\
    \hline
    [MOBILE] Detale usera z rankingu & Jakub \\
    \hline
    [MOBILE] Spiąć "My Decks" z API & Daniel \\
    \hline
    [MOBILE] Spiąć "Create Decks" z API & Daniel \\
    \hline
    [MOBILE] Dodać ekran podglądu fiszek & Daniel \\
    \hline
    [MOBILE] Dodać ekran podglądu decku & Daniel \\
    \hline
    [WEB] Ranking użytkowników & Oliwier \\
    \hline
    [MOBILE] Dodać listę udostępnionych talii & Jakub \\
    \hline
    [MOBILE] Spiąć podgląd decku z API & Daniel \\
    \hline
    [MOBILE] Dodać ekran tworzenia fiszki & Daniel \\
    \hline
    [MOBILE] Spiąć ekran tworzenia fiszki z API & Daniel \\
    \hline
    [MOBILE] Dodać service flashcards & Daniel \\
    \hline
    [MOBILE] Utworzyć component pobierania danych z API & Daniel \\
    \hline
    [MOBILE] Dodać edycję i usunięcie fiszki & Daniel \\
    \hline
    [MOBILE] Dodać ekran settings dla decku & Daniel \\
    \hline
    [WEB] Poprawa rankingów & Oliwier \\
    \hline
    [MOBILE] Dodać i spiąć memorized flashcards & Daniel \\
    \hline
    [BACKEND] Testy integracyjne & Oliwier \\
    \hline
    [WEB] Utworzenie strony public decks & Oliwier \\
    \hline
    [WEB] Dodanie obsługi złożonych komend głosowych & Oliwier \\
    \hline
    [BACKEND] Utworzenie modelu nlp do rozumienia semantyki słów & Oliwier \\
    \hline
    [MOBILE] Podczas wyszukiwania decku input traci focus & Jakub \\
    \hline
    [MOBILE] Użyć use refa w search public decks, users i reports & Jakub \\
    \hline
    [MOBILE] Dodać modal z instrukcją sterowania głosu & Jakub \\
    \hline
    [MOBILE] Dodać alerty po responsie & Jakub \\
    \hline
    [MOBILE] Poprawić animacje & Jakub \\
    \hline
    [MOBILE] Naprawić błąd nawigacji dla statystyk & Jakub \\
    \hline
    [MOBILE] Poprawić responsywność widoku public decks dla tabletu & Jakub \\
    \hline
    [MOBILE] Dodać panel dla moderatora & Jakub \\
    \hline
    [MOBILE] Dodać ikonę moderatora do pasku nawigacji & Jakub \\
    \hline
    [MOBILE] Naprawić query search w deck list i user stats & Jakub \\
    \hline
    [MOBILE] Spiąć "Create Decks" z API & Jakub \\
    \hline
    [MOBILE] Dodać ekran podglądu fiszek & Jakub \\
    \hline
    [MOBILE] Dodać ekran podglądu decku & Jakub \\
    \hline
    [WEB] Utworzenie strony public decks & Oliwier \\
    \hline
    [WEB] Obsługa błędów (rejestracja/logowanie) & Wiktor \\
    \hline
    [MOBILE] Spiąć generowanie treści fiszki z AI & Daniel \\
    \hline
    [MOBILE] Utworzyć i spiąć tryb nauki & Daniel \\
    \hline
    [MOBILE] Spiąć ekran tworzenia fiszki z API & Daniel \\
    \hline
    [MOBILE] Dodać odsłuchanie fiszek w memorized i unmemorized flashcards & Daniel \\
    \hline
    [BACKEND] Dodanie tabeli i endpointów dla zgłoszonych talii & Oliwier \\
    \hline
    [WEB] Utworzenie panelu moderatora & Oliwier \\
    \hline
    [BACKEND] Usunięcie usera usuwa wszystkie powiązane rekordy, dodanie endpointu do usuwania innych userów & Jakub \\
    \hline
    [WEB] Poprawa okienka do edycji nazwy decku & Oliwier \\
    \hline
    [WEB] Poprawić responsywność strony & Oliwier \\
    \hline
    [SERWER] Zainicjować maszynę wirtualną w Azure & Daniel \\
    \hline
    [SERWER] Skonfigurować użytkowników & Daniel \\
    \hline
    [SERWER] Skonfigurować sieć, środowisko API, bazę danych i aplikację web & Daniel \\
    \hline
    [MOBILE] Dodać sterowanie głosem & Jakub \\
    \hline
    [WEB] Dodać usuwanie użytkownika do panelu moderatora & Oliwier \\
    \hline
    [BACKEND] Poprawa modelu do sterowania głosem & Oliwier \\
    \hline
    [MOBILE] Dodać podgląd i pobranie decku publicznego & Daniel \\
    \hline
    [WEB] Dodanie instrukcji sterowania głosem & Oliwier \\
    \hline
    [MOBILE] Dodać wyszukiwanie swoich decków & Daniel \\
    \hline
    [BACKEND] Dodać biblioteki do requirements & Oliwier \\
    \hline
    [MOBILE] Poprawić udostępnianie decku & Daniel \\
    \hline
    [SERWER] Skonfigurować połączenie HTTPS i domenę & Daniel \\
    \hline
    [MOBILE] Dodać potwierdzenie czy użytkownik akceptuje treść wygenerowaną przez ChatGPT & Daniel \\
    \hline
    [MOBILE] Spiąć z API zgłaszanie decku & Daniel \\
    \hline
                        \caption{Zadania wykonane w sprincie 7.}
\end{longtable}

