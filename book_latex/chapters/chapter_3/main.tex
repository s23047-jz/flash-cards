\chapter{Projekt w kontekście problemu.}

\section{Omówienie zakresu projektu.}

W odpowiedzi na zdefiniowany i omówiony w Rozdziale II problem, niniejszy projekt zakłada wytworzenie aplikacji edukacyjnej, która umożliwi naukę metodą wykorzystującą fiszki. Projekt obejmuje dostarczenie aplikacji webowej dostępnej z poziomu przeglądarki internetowej oraz aplikacji mobilnej dla urządzeń z systemem android lub IOS. W zakres prac wlicza się także zbudowanie pełnej infrastruktury wspierającej aplikację - backend oraz serwer utrzymujący cały system

\section{Proponowane rozwiązanie.}

Projekt bazuje na kilku kluczowych założeniach, które mają na celu bezpośrednie rozwiązanie problemów opisanych w Rozdziale II:

\begin{itemize}
    \item Obsługa Głosowa: integracja obsługi głosowej aplikacji ma ułatwić korzystanie z niej w sytuacjach w których użytkownik ma ograniczone możliwości fizycznej obsługi urządzenia. Dzięki temu nauka jest możliwa w każdych warunkach, co odpowiada na problem ograniczonej dostępności tradycyjnych metod nauki.
    \item Narzędzie AI do tworzenia fiszek: udostępnienie narzędzia umożliwiającego automatyczne generowanie i dobieranie definicji/odpowiedzi do zagadnień w fiszkach. Pozwala przyśpieszyć i uprzystępnić proces redagowania fiszek.
    \end{itemize}
