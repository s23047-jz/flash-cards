\chapter{Społeczny aspekt projektu}

\section{Gamifikacja w kontekście produktu}
Została przeprowadzona analiza związana z wpływem rywalizacji na aktywność użytkownika, wynika z niej, że elementy rankingowe mają wpływ na zwiększoną aktywność użytkowników, z tego powodu aplikacja zawiera ranking popularności użytkowników i talii.\cite{ablyGamification} Pozycja w rankingu użytkownika jest wyliczana na podstawie sumy wszystkich pobrań talii upublicznionych dla innych odbiorców systemu. Ranking talii to zestawy fiszek o największej liczbę pobrań. System rankingowy ma na celu wzbudzenie rywalizacji u użytkowników, co ma ich sprowokować do większej aktywności. System punktów zachęca także odbiorców systemu do tworzenia nowych zestawów talii i dzielenia się z nimi w celu zwiększenia swojej szansy na poprawienie swojej pozycji rankingowej.


\section{Ryzyko związane z nauką przy pomocy systemu Fishki}
System fiszki ma możliwość generowania definicji na podstawie słowa, wykorzystując Chat GPT, taka funkcjonalność niesie ze sobą różne zagrożenia związane z rozwojem umysłowym odbiorcy. Korzystanie z generowania treści może doprowadzić do rozleniwienia użytkownika. Odbiorca, wykorzystując Chat GPT do tworzenia definicji może nie podejmować żadnej refleksji, czy dostarczona treść ma jakikolwiek sens. W przypadku, w którym użytkownik nie jest zaznajomiony z tematem lub zagadnieniami może doprowadzić do sytuacji, w której wygenerowana treść będzie niepoprawna, ale odbiorca ze względu na brak wiedzy nie będzie tego świadomy. Chat GPT potrafi generować nieprawdziwe informacje co może spowodować nauczenie się przez użytkownika zmyślonych treści.Warto też podkreślić, że informacje przekazywane do czatu są wykorzystywane do uczenia modeli, jeżeli użytkownik przy generowaniu treści wykorzysta dane wrażliwe trafią one do twórców narzędzia.\cite{chatGptRisk} Sytuacja, gdy użytkownik jest zaznajomiony z materiałem do nauki i jest w stanie ocenić jakość generowanych treści też niekoniecznie musi być lepsza, ponieważ odbiorca, tworząc samemu treści fiszek utrwala sobie zagadnienia. Bezmyślne zapamiętywanie informacji to kolejne zagrożenie, jakie może pojawić się w czasie nauki nowego materiału. Uczenie się na pamięć treści bez dogłębnego zrozumienia ich, sprawia, że zagadnienie jest krócej pamiętane przez użytkownika, a także doprowadza do sytuacji, w której użytkownik nie rozumie i nie wie jak zastosować przyswojone informacje. Powyższe problemy występuję także w momencie, w którym użytkownik korzysta z zestawu fiszek utworzonych przez kogoś innego.
