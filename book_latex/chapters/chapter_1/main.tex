\chapter{Wstęp}

\section{O projekcie}\label{ch:wstep}

Niniejszy projekt i praca podejmuje temat wytworzenia multiplatformowej aplikacji przeznaczonej do nauki metodą fiszek. Głównym celem projektu jest usprawnienie procesu nauki poprzez wykorzystanie nowoczesnych technologii.

Geneza projektu wywodzi się ze środowiska zespołu projektowego oraz otoczenia uczelnianego, \\bezpośredniego zapotrzebowania na aplikacje, która spełni wszystkie wymagane funkcjonalności, co skutkować będzie tym, iż proces nauki przez powtórki będzie skuteczny i wygodny. Te funkcjonalności oraz właściwości zostały zdefiniowane początkowo jako pomysł z osobistych doświadczeń członków zespołu z alternatywnymi aplikacjami. Później na podstawie przeprowadzonej w niniejszej pracy analizie konkurencyjnych rozwiązań, pomysł przeobraził się w ostateczną wizję produktu.

Wśród wielu aplikacji przeznaczonych do nauki przy pomocy różnych metod, ich wspólnym mianownikiem są regularne powtórki. Zdefiniowanie wymagań dla aplikacji tego typu nie jest możliwe bez dobrze sprecyzowanej problematyki związanej z zagadnieniem utrwalania wiedzy. Problematyka ta została omówiona w oparciu o badania podejmujące temat systematyki w nauce. W następnym rozdziale pracy przyjrzymy się bardziej szczegółowo kwestii związanej z tym, czy regularne powtarzanie materiału jest skuteczne i jak nowoczesne technologie mogą ten proces usprawnić.

Niniejsza praca także podsumowuje realizację podjętego projektu. Opisuje szczegółowo m.in. podjętą metodykę, organizację, podział i nakład pracy, analizę konkurencji, zdefiniowane wymagania, architekturę projektu, użyte technologie, implementację techniczną oraz testy.

Realizowany projekt ma na celu nie tylko zbudowanie użytecznej aplikacji, ale także przyczynienie się do zrozumienia, jak nowoczesne technologie mogą wspierać edukację i w jakim kierunku należy rozwijać dostępne narzędzia. Dlatego też obowiązkowo podejmujemy temat dalszego rozwoju oraz przyszłości wdrożonego systemu.
