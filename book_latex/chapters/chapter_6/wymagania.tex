\setstretch{1}
\section{Udziałowcy}

    \begin{stakeholder}[caption={Zespół projektowy jako udziałowiec.}],
    \id{UOB 01}
    \name{Zespół projektowy}
    \descr{Zespół wytwarzający system}
    \type{Ożywiony bezpośredni}
    \viewpoint{Wytworzenie aplikacji mobilnej, strony internetowej, zaplecza technicznego systemu; zarządzanie, utrzymanie i rozwój systemu}
    \limitations{Czynniki ludzkie organizacyjne, deadline i czas}
    \requ{WF 01, WF 02, WF 09}
\end{stakeholder}

Administrator - zarządza utrzymaniem systemu i jego środowiska technicznego, ma dostęp do kodu źródłowego oraz serwerów systemu, odpowiada za bazy danych oraz ciągłość funkcji.

\begin{stakeholder}[caption={Administrator systemu jako udziałowiec.}]
    \id{UOB 02}
    \name{Administrator systemu}
    \descr{Administrator wytworzonego systemu projektowego}
    \type{Ożywiony bezpośredni}
    \viewpoint{Nadzór i techniczne utrzymanie systemu, administracja środowiskiem systemowym}
    \limitations{Koszty utrzymaniowe - budżet chmury, parametry techniczne środowiska systemowego}
    \requ{ŚD 04, WN03}
\end{stakeholder}

Użytkownik systemu - osoba używająca systemu oraz wszystkich wytworzonych funkcjonalności użytkowych. Brak narzuconych ograniczeń ilościowych, prototyp projektu zakłada wstępną obsługę maksymalnie kilkuset użytkowników (200 - 300).

\begin{stakeholder}[caption={Użytkownik systemu jako udziałowiec.}]
    \id{UOB 03}
    \name{Użytkownik systemu}
    \descr{Osoba korzystająca z systemu za pośrednictwem aplikacji mobilnej lub strony web}
    \type{Ożywiony bezpośredni}
    \viewpoint{Korzystanie z systemu: tworzenie talii fiszek, wykorzystywanie ich do nauki, udostępnianie talii innym}
    \limitations{Wytworzone funkcjonalności}
    \requ{WF 03, WF 04, WF 05,  WF 06,  WF 07,  WF 08, WF 10, WF 11}
\end{stakeholder}

\begin{stakeholder}[caption={Dostawca technologii chmury jako udziałowiec.}]
    \id{UOB 04}
    \name{Dostawca technologii chmury}
    \descr{Firma świadcząca usługi chmurowe}
    \type{Ożywiony bezpośredni}
    \viewpoint{Dostarczeczenie infrastruktury i środowiska technicznego w którym działa system projektowy.}
    \limitations{Budżet projektowy, parametry techniczne środowiska}
    \requ{ŚD 04}
\end{stakeholder}

    \section{Wymagania ogólne}

\subsection{Moduł autoryzacji}

    \begin{requirementstab}[label={tab:requirements:general},caption={Przykładowe wymaganie ogólne lub dziedzinowe}]
    \id{ WO1 }
    \priority{ M – must }
    \name{ Moduł autoryzacji }
    \descr{Rejestracja konta nowego użytkownika.
Możliwość usunięcia konta użytkownika.
Logowanie do systemu. Wylogowanie użytkownika z systemu.
Edycja danych użytkownika.
}
    \sholder{Zespół projektowy (UOB 01) \newline
Użytkownik systemu (UOB 03)
}
    \reqrelated{Rejestracja konta użytkownika (WF 01) \newline
Logowanie do systemu (WF 02) \newline
Wylogowanie z systemu (WF 03) \newline
Edycja danych użytkownika (WF 04) \newline
Usunięcie konta użytkownika (WF 05) \newline
Weryfikacja konta użytkownika poprzez e-mail  (WF 14)
}

\end{requirementstab}

    \subsection{Moduł zarządzania talią}

\begin{requirementstab}[label={tab:requirements:deck},caption={Wymaganie ogólne dla modułu zarządzania talią}]
    \id{ WO2 }
    \priority{ M – must }
    \name{ Moduł zarządzania talią }
    \descr{Tworzenie talii fiszek. Pobranie talii utworzonej przez innych użytkowników. Edycja utworzonych talii fiszek lub edycja talii pobranej przez innych użytkowników. Usunięcie talii fiszek.}
    \sholder{Zespół projektowy (UOB 01) \newline
Użytkownik systemu (UOB 03)
}
    \reqrelated{Tworzenie talii fiszek (WF 06) \newline
Usunięcie talii fiszek (WF 07) \newline
Edycja talii fiszek (WF 08) \newline
Import talii innych użytkowników (WF 11)
}

\end{requirementstab}

    \subsection{Moduł uczenia}

\begin{requirementstab}[label={tab:requirements:learning},caption={Wymaganie ogólne dla modułu uczenia}]
    \id{ W 03 }
    \priority{ M – must }
    \name{ Moduł uczenia }
    \descr{Tryb sterowania głosem pozwala na uruchomienie talii fiszek w specjalnym trybie, który pozwala na sterowanie talią fiszek przy wykorzystaniu komend głosowych. Zwykły tryb uczenia uruchamia talie w pełnym ekranie i pozwala na podzielenie talii na fiszki, które użytkownik zapamiętał i na te niezapamiętane.}
    \sholder{Zespół projektowy (UOB 01) \newline
Użytkownik systemu (UOB 03)
}
    \reqrelated{Tryb uczenia się z talii fiszek (WF 09) \newline
Sterowanie talią przy użyciu mowy (WF 10)
}

\end{requirementstab}

    \subsection{Moduł udogodnień}

\begin{requirementstab}[label={tab:requirements:conveniences},caption={Wymaganie ogólne dla modułu udogodnień}]
    \id{ W 04 }
    \priority{ S – should ) }
    \name{ Moduł udogodnień }
    \descr{Tryb dark mode i light mode pozwala użytkownikowi na zmianę kolorystyki systemy w celu uniknięcia przemęczenia wzroku. Kalendarz oznacza dni, w których użytkownik korzysta z aplikacji, może to spowodować większą motywację u użytkownika aby regularnie korzystał z systemu.}
    \sholder{Zespół projektowy (UOB 01) \newline
Użytkownik systemu (UOB 03)
}
    \reqrelated{Tryb dark mode i light mode (WF 12) \newline
Kalendarz śledzący aktywność (WF 13)
}

\end{requirementstab}

\subsection{Moduł wspomagania tworzenia talii}

\begin{requirementstab}[label={tab:requirements:deck_creation_support},caption={Wymaganie ogólne dla modułu wspomagania tworzenia talii}]
    \id{ W 05 }
    \priority{ W – won’t }
    \name{ Moduł wspomagania tworzenia talii }
    \descr{Analiza dokumentów poprzez wykorzystanie sztucznej inteligencji pozwala na wygenerowanie zestawu fiszek poprzez wgranie do systemu dokumentu w formacie csv lub pdf. Użytkownik może wygenerować treść fiszki, podając kilka słów kluczowych na podstawie których sztuczna inteligencja przeszukuje bazę definicji i na podstawie wyszukanych definicji generuje treść.}
    \sholder{Użytkownik systemu (UOB 03)}
    \reqrelated{Tworzenie talii fiszek przez analizę dokumentu (WF 15) \newline
Generowanie treści fiszki na podstawie słów kluczowych (WF 16)
}

\end{requirementstab}

    \section{Wymagania funkcjonalne}

\subsection{Rejestracja konta użytkownika}

    \begin{requirementstab}[label={tab:requirements:user_registration},caption={Wymaganie funkcjonalne dla procesu rejestracji użytkownika}]
    \id{WF 01}
    \priority{M – must}
    \name{Rejestracja nowego użytkownika w systemie}
    \descr{Użytkownik musi utworzyć konto aby móc korzystać z aplikacji.}
    \acceptcrit{Nowy użytkownik dodany do systemu}
    \inputdata{nickname, e-mail, hasło}
    \preconditions{Użytkownik musi posiadać adres e-mail}
    \postconditions{Utworzenie konta użytkownika}
    \exceptions{Brak łączności z bazą danych. Wprowadzenie dane przez użytkownika istnieją już w bazie danych.}
    \implementation{Punkt 7.3.1}
    \sholder{Zespół projektowy (UOB 01)
Użytkownik systemu (UOB 03) }
    \reqrelated{Logowanie do systemu (WF 02) \newline
Wylogowanie z systemu (WF 03) \newline
Edycja danych użytkownika (WF 04) \newline
Usunięcie konta użytkownika (WF 05) \newline
Weryfikacja konta użytkownika poprzez e-mail (WF 14)}
\end{requirementstab}

    \subsection{Logowanie do systemu}

\begin{requirementstab}[label={tab:requirements:system_login},caption={Wymaganie funkcjonalne dla procesu logowania do systemu}]
    \id{WF 02}
    \priority{M – must}
    \name{Logowanie do systemu}
    \descr{Użytkownik musi zalogować się do systemu, aby mieć dostęp do systemu.}
    \acceptcrit{Pomyślne zalogowanie do systemu}
    \inputdata{e-mail i hasło}
    \preconditions{Posiadane konto użytkownika, konto musi być aktywne.}
    \postconditions{Dostęp do systemu}
    \exceptions{Brak łączności z bazą danych. Wprowadzenie niepoprawnych danych logowania, co uniemożliwia zalogowanie do systemu.}
    \implementation{Punkt 7.3.2}
    \sholder{Zespół projektowy (UOB 01) }
    \reqrelated{Rejestracja konta użytkownika (WF 01) \newline
Wylogowanie z systemu (WF 03) \newline
Edycja danych użytkownika (WF 04) \newline
Usunięcie konta użytkownika (WF 05) \newline
Weryfikacja konta użytkownika poprzez e-mail (WF 14)}
\end{requirementstab}

    \subsection{Wylogowanie z systemu}

\begin{requirementstab}[label={tab:requirements:system_logout},caption={Wymaganie funkcjonalne dla procesu wylogowania z systemu}]
    \id{WF 03}
    \priority{M – must}
    \name{Wylogowanie z systemu}
    \descr{Jako użytkownik chciałbym mieć możliwość wylogowania z systemu, aby inne osoby nie mogły korzystać z mojego konta.}
    \acceptcrit{Wylogowanie użytkownika z systemu.}
    \inputdata{Brak}
    \preconditions{Użytkownik zalogowany do systemu}
    \postconditions{Wylogowanie z systemu}
    \exceptions{Awaria bazy danych lub brak połączenia z bazą.}
    \implementation{Punkt 7.3.3}
    \sholder{Użytkownik systemu (UOB 03) }
    \reqrelated{Rejestracja konta użytkownika (WF 01) \newline
Logowanie do systemu (WF 02) \newline
Edycja danych użytkownika (WF 04) \newline
Usunięcie konta użytkownika (WF 05) \newline
Weryfikacja konta użytkownika poprzez e-mail (WF 14)}
\end{requirementstab}

    \subsection{Edycja danych użytkownika}

\begin{requirementstab}[label={tab:requirements:user_data_edit},caption={Wymaganie funkcjonalne dla procesu edycji danych użytkownika}]
    \id{WF 04}
    \priority{M – must}
    \name{Edycja danych użytkownika}
    \descr{Jako użytkownik muszę mieć możliwość zmiany hasła lub e-mail ponieważ mogę stracić dostęp do konta e-mail lub moje hasło z różnych powodów może stać się jawne.}
    \acceptcrit{Zmienione parametry logowania}
    \inputdata{Użytkownik podaje nowy parametr wraz z hasłem, w celu edycji danych użytkownika}
    \preconditions{Posiadane konto użytkownika, użytkownik zalogowany do systemu.}
    \postconditions{Zmienione parametry logowania}
    \exceptions{Awaria bazy danych lub brak połączenia z bazą. Wprowadzenie niepoprawnych danych co uniemożliwia zmianę parametrów użytkownika.}
    \implementation{Punkt 7.3.4}
    \sholder{Użytkownik systemu (UOB 03) }
    \reqrelated{Rejestracja konta użytkownika (WF 01) \newline
Logowanie do systemu (WF 02) \newline
Wylogowanie z systemu (WF 03) \newline
Usunięcie konta użytkownika (WF 05) \newline
Weryfikacja konta użytkownika poprzez e-mail (WF 14)}
\end{requirementstab}

    \subsection{Usunięcie konta użytkownika}

\begin{requirementstab}[label={tab:requirements:user_account_deletion},caption={Wymaganie funkcjonalne dla procesu usunięcia konta użytkownika}]
    \id{WF 05}
    \priority{M – must}
    \name{Usunięcie konta użytkownika}
    \descr{Jako użytkownik chcę mieć możliwość usunięcia swojego konta, gdy przestanę korzystać z aplikacji.}
    \acceptcrit{Usunięte konto użytkownika}
    \inputdata{e-mail i hasło}
    \preconditions{Posiadane konto użytkownika, użytkownik zalogowany do systemu.}
    \postconditions{Konto użytkownika usunięte z systemu}
    \exceptions{Awaria bazy danych lub brak połączenia z bazą. Wprowadzenie niepoprawnych danych, co uniemożliwia usunięcie konta użytkownika.}
    \implementation{Punkt 7.3.5}
    \sholder{Użytkownik systemu (UOB 03) }
    \reqrelated{Rejestracja konta użytkownika (WF 01) \newline
Logowanie do systemu (WF 02) \newline
Wylogowanie z systemu (WF 03) \newline
Edycja danych użytkownika (WF 04) \newline
Weryfikacja konta użytkownika poprzez e-mail (WF 14)}
\end{requirementstab}

    \subsection{Tworzenie talii fiszek}

\begin{requirementstab}[label={tab:requirements:deck_creation},caption={Wymaganie funkcjonalne dla procesu tworzenia talii fiszek}]
    \id{WF 06}
    \priority{M – must}
    \name{Tworzenie talii fiszek}
    \descr{Jako użytkownik muszę mieć możliwość utworzenia własnej talii fiszek, w celu uczenia się zagadnień, które mają dla mnie znaczenie.}
    \acceptcrit{Utworzona talia fiszek}
    \inputdata{tytuł talii fiszek, kategoria talii, tytuł fiszki, treść fiszki}
    \preconditions{Posiadane konto użytkownika, użytkownik zalogowany do systemu.}
    \postconditions{Utworzona nowa talia fiszek.}
    \exceptions{Awaria bazy danych lub brak połączenia z bazą. Puste pole z nazwą talii uniemożliwia tworzenie talii. Pusty tytuł fiszki lub pusta treść uniemożliwia utworzenie fiszki.}
    \implementation{Punkt 7.3.6}
    \sholder{Użytkownik systemu (UOB 03) }
    \reqrelated{Usunięcie talii fiszek (WF 07) \newline
Edycja talii fiszek (WF 08) \newline
Import talii innych użytkowników (WF 11)}
\end{requirementstab}

\subsection{Usunięcie talii fiszek}

\begin{requirementstab}[label={tab:requirements:deck_deletion},caption={Wymaganie funkcjonalne dla procesu usunięcia talii fiszek}]
    \id{WF 07}
    \priority{M – must}
    \name{Usunięcie talii fiszek}
    \descr{Jako użytkownik chcę mieć możliwość usunięcia talii fiszek, z których nie będę już korzystał.}
    \acceptcrit{Usunięta talia fiszek}
    \inputdata{Brak}
    \preconditions{Posiadane konto użytkownika, użytkownik zalogowany do systemu. Utworzona talia fiszek, którą można usunąć.}
    \postconditions{Talia fiszek zostaje usunięta.}
    \exceptions{Awaria bazy danych lub brak połączenia z bazą.}
    \implementation{Punkt 7.3.7}
    \sholder{Użytkownik systemu (UOB 03) }
    \reqrelated{Tworzenie talii fiszek (WF 06) \newline
Edycja talii fiszek (WF 08) \newline
Import talii innych użytkowników (WF 11)}
\end{requirementstab}

    \subsection{Edycja talii fiszek}

\begin{requirementstab}[label={tab:requirements:deck_edit},caption={Wymaganie funkcjonalne dla procesu edycji talii fiszek}]
    \id{WF 08}
    \priority{M – must}
    \name{Edycja talii fiszek}
    \descr{Jako użytkownik chcę mieć możliwość edycji talii fiszek, aby zaktualizować informacje dotyczące talii.}
    \acceptcrit{Zaktualizowana zawartość talii}
    \inputdata{Nowy tytuł talii, kategoria lub zmieniona treść fiszki.}
    \preconditions{Talia fiszek, która będzie edytowana.}
    \postconditions{Talia fiszek zostaje zaktualizowana o nowe informacje.}
    \exceptions{Awaria bazy danych lub brak połączenia z bazą. Puste pole dotyczące tytułu talii uniemożliwia zapisanie zmian.}
    \implementation{Punkt 7.3.8}
    \sholder{Użytkownik systemu (UOB 03) }
    \reqrelated{Tworzenie talii fiszek (WF 06) \newline
Usunięcie talii fiszek (WF 07) \newline
Import talii innych użytkowników (WF 11)}
\end{requirementstab}

    \subsection{Tryb uczenia się z talii fiszek}

\begin{requirementstab}[label={tab:requirements:learning_mode},caption={Wymaganie funkcjonalne dla trybu uczenia się z talii fiszek}]
    \id{WF 09}
    \priority{M – must}
    \name{Tryb uczenia się z talii fiszek}
    \descr{System musi posiadać tryb uczenia się, który pozwoli użytkownikowi na naukę i zapamiętywanie zagadnień znajdujących się w talii fiszek. W trakcie nauki użytkownik dzieli talię na zestaw z pojęciami które już opanował i na te, których jeszcze nie pamięta.}
    \acceptcrit{Talia fiszek zostaje podzielona na dwa zbiory, fiszki zapamiętane i niezapamiętane.}
    \inputdata{Brak}
    \preconditions{Utworzona talia fiszek lub pobrana talia fiszek, która zostanie wykorzystana do nauki.}
    \postconditions{Talia fiszek zostaje podzielona na dwa zbiory, fiszki zapamiętane i niezapamiętane.}
    \exceptions{Awaria bazy danych lub brak połączenia z bazą.}
    \implementation{Punkt 7.3.9}
    \sholder{Zespół projektowy (UOB 01) }
    \reqrelated{Sterowanie talią przy użyciu mowy (WF 10)}
\end{requirementstab}

    \subsection{Sterowanie talią przy użyciu mowy}

\begin{requirementstab}[label={tab:requirements:voice_control},caption={Wymaganie funkcjonalne dla sterowania talią przy użyciu mowy}]
    \id{WF 10}
    \priority{M – must}
    \name{Sterowanie talią przy użyciu mowy}
    \descr{Jako użytkownik chcę mieć możliwość uczenia się w warunkach, w których odczytywanie treści fiszki jest utrudnione, na przykład w trakcie prowadzenia samochodu. Sterowanie talią z przy użyciu mowy i odczytanie zawartości fiszki przez sztuczną inteligencję pozwala na naukę podczas prowadzenia pojazdu.}
    \acceptcrit{Użytkownik steruje talią fiszek przy użyciu komend głosowych}
    \inputdata{Brak}
    \preconditions{Utworzona talia fiszek}
    \postconditions{Talia fiszek pozostaje niezmieniona}
    \exceptions{Awaria bazy danych lub brak połączenia z bazą danych. Hałas, który aplikacja może przechwytywać, przez co komendy głosowe mogą działać niepoprawnie. Fiszki utworzone w innym języku niż angielski co może spowodować trudności w odczytaniu ich zawartości przez syntezator mowy.}
    \implementation{Punkt 7.3.10}
    \sholder{Użytkownik systemu (UOB 03) }
    \reqrelated{Tryb uczenia się z talii fiszek (WF 09)}
\end{requirementstab}

    \subsection{Import talii innych użytkowników}

\begin{requirementstab}[label={tab:requirements:deck_import},caption={Wymaganie funkcjonalne dla procesu importowania talii fiszek od innych użytkowników}]
    \id{WF 11}
    \priority{M – must}
    \name{Import talii innych użytkowników}
    \descr{Jako użytkownik chcę mieć możliwość pobrania talii od innych użytkowników, aby mieć łatwy dostęp do treści i zagadnień, które mnie interesują.}
    \acceptcrit{Talia dodana do zakładki talii pobranych od innych użytkowników.}
    \inputdata{Brak}
    \preconditions{Talia którą użytkownik może pobrać.}
    \postconditions{Talia dodana do zakładki talii pobranych od innych użytkowników.}
    \exceptions{Awaria bazy danych lub brak połączenia z bazą danych.}
    \implementation{Punkt 7.3.11}
    \sholder{Użytkownik systemu (UOB 03) }
    \reqrelated{Tworzenie talii fiszek (WF 06) \newline
Usunięcie talii fiszek (WF 07) \newline
Edycja talii fiszek (WF 08)}
\end{requirementstab}

    \subsection{Tryb dark mode i light mode}

\begin{requirementstab}[label={tab:requirements:dark_light_mode},caption={Wymaganie funkcjonalne dla trybu dark mode i light mode}]
    \id{WF 12}
    \priority{S – should}
    \name{Tryb dark mode i light mode}
    \descr{Jako użytkownik chciałbym mieć możliwość zmiany kontrolowania jasności systemu aby mój wzrok się nie przemęczał.}
    \acceptcrit{Zmiana koloru interfejsu aplikacji.}
    \inputdata{Brak}
    \preconditions{Posiadane konto użytkownika, użytkownik zalogowany do systemu.}
    \postconditions{Zmiana koloru interfejsu aplikacji.}
    \exceptions{Awaria bazy danych lub brak połączenia z bazą danych.}
    \implementation{Punkt 7.3.12}
    \sholder{Użytkownik systemu (UOB 03) }
    \reqrelated{Kalendarz śledzący aktywność (WF 13)}
\end{requirementstab}

    \subsection{Kalendarz śledzący aktywność}

\begin{requirementstab}[label={tab:requirements:activity_calendar},caption={Wymaganie funkcjonalne dla kalendarza śledzącego aktywność użytkownika}]
    \id{WF 13}
    \priority{S – should}
    \name{Kalendarz śledzący aktywność}
    \descr{Kalendarz odznaczający dni, w których użytkownik korzystał z aplikacji, mógłby zwiększyć motywację użytkownika do regularnego korzystania z aplikacji.}
    \acceptcrit{Kalendarz oznacza dzień w momencie uruchomienia przez użytkownika aplikacji.}
    \inputdata{Brak}
    \preconditions{Posiadane konto użytkownika, użytkownik zalogowany do systemu.}
    \postconditions{Dzień oznaczony w kalendarzu.}
    \exceptions{Awaria bazy danych lub brak połączenia z bazą danych.}
    \implementation{ Brak }
    \sholder{Zespół projektowy (UOB 01) }
    \reqrelated{Tryb dark mode i light mode (WF 12)}
\end{requirementstab}

    \subsection{Weryfikacja konta użytkownika poprzez e-mail}

\begin{requirementstab}[label={tab:requirements:e-mail_verification},caption={Wymaganie funkcjonalne dla weryfikacji konta użytkownika poprzez e-mail}]
    \id{WF 14}
    \priority{C – could}
    \name{Weryfikacja konta użytkownika poprzez e-mail}
    \descr{Użytkownik po pomyślnej rejestracji, otrzyma na e-mail wiadomość z linkiem przekierowującym na stronę webową, gdzie nastąpi wysłanie żądania o aktywację konta użytkownika.}
    \acceptcrit{Wiadomość dostarczona na wskazany e-mail, poprawnie działająca aktywacja konta, poprawne przekierowanie.}
    \inputdata{Token autoryzujący}
    \preconditions{Użytkownik musi posiadać istniejący adres e-mail, na który przyjdzie wiadomość z linkiem aktywującym konto.}
    \postconditions{Użytkownik aktywuje konto.}
    \exceptions{Awaria bazy danych lub brak połączenia z bazą danych. Użytkownik nie ma dostępu do podanego przez siebie adresu e-mail lub podany adres jest błędny. Wiadomość nie dotarła do użytkownika.}
    \implementation{ Brak }
    \sholder{Zespół projektowy (UOB 01) }
    \reqrelated{Rejestracja konta użytkownika (WF 01) \newline
Logowanie do systemu (WF 02)}
\end{requirementstab}

    \subsection{Tworzenie talii fiszek poprzez zeskanowanie dokumentu}

\begin{requirementstab}[label={tab:requirements:deck_creation_scan},caption={Wymaganie funkcjonalne dla tworzenia talii fiszek poprzez zeskanowanie dokumentu}]
    \id{WF 15}
    \priority{W – won't}
    \name{Tworzenie talii fiszek poprzez zeskanowanie dokumentu}
    \descr{Jako użytkownik chciałbym mieć możliwość szybkiego utworzenia zestawu talii fiszek poprzez wgranie gotowego dokumentu w formacie csv lub pdf z którego zostałaby utworzona talia fiszek na podstawie zagadnień znajdujących się w dokumencie.}
    \acceptcrit{Talia fiszek utworzona po analizie dokumentu.}
    \inputdata{Dokument do analizy w formacie csv lub pdf.}
    \preconditions{Posiadane konto użytkownika, użytkownik zalogowany do systemu, dokument do analizy.}
    \postconditions{Utworzona talia fiszek.}
    \exceptions{Awaria bazy danych lub brak połączenia z bazą danych. Dokument nie zdatny do odczytu.}
    \implementation{ Brak }
    \sholder{Użytkownik systemu (UOB 03) }
    \reqrelated{Generowanie treści fiszki na podstawie słów kluczowych (WF 16)}
\end{requirementstab}

    \subsection{Generowanie treści fiszki na podstawie słów kluczowych}

\begin{requirementstab}[label={tab:requirements:content_generation_keywords},caption={Wymaganie funkcjonalne dla generowania treści fiszki na podstawie słów kluczowych}]
    \id{WF 16}
    \priority{M – must}
    \name{Generowanie treści fiszki na podstawie słów kluczowych}
    \descr{Jako użytkownik chcę, aby system potrafił generować definicje zagadnień, które wpisałem na pierwszą stronę karty, co ułatwi tworzenie talii.}
    \acceptcrit{Zawartość fiszki wygenerowana przy pomocy zewnętrznego API na podstawie podanej treści na stronie polu dla przedniej strony fiszki.}
    \inputdata{Brak}
    \preconditions{Utworzone konto użytkownika.}
    \postconditions{Treść fiszki wygenerowana na podstawie zawartości wpisanej w pole przedniej strony karty.}
    \exceptions{Awaria API lub połączenia z API. Podanie przez użytkownika zdania nie mającego sensu.}
    \implementation{ Punkt 7.3.13 }
    \sholder{Użytkownik systemu (UOB 03) }
    \reqrelated{Tworzenie talii fiszek przez analizę dokumentu (WF 15)}
\end{requirementstab}

    \subsection{Resetowanie hasła na konta}

\begin{requirementstab}[label={tab:requirements:password_reset},caption={Wymaganie funkcjonalne dla resetowania hasła}]
    \id{WF 017}
    \priority{S – should}
    \name{Resetowanie hasła na konta}
    \descr{Użytkownik po podaniu adresu mailowego połączonego z kontem, otrzyma na e-mail wiadomość z linkiem przekierowującym na stronę webową, gdzie będzie mógł podać nowe hasło.}
    \acceptcrit{Pomyślna zmiana hasła.}
    \inputdata{Token autoryzujący, nowe hasło}
    \preconditions{Użytkownik musi posiadać istniejący adres e-mail, na który przyjdzie wiadomość z linkiem przekierowującym na widok strony internetowej.}
    \postconditions{Użytkownik pomyślnie zmienia hasło.}
    \exceptions{Awaria bazy danych lub brak połączenia z bazą danych. Użytkownik nie ma dostępu do podanego przez siebie adresu e-mail lub podany adres jest błędny. Wiadomość nie dotarła do użytkownika.}
    \implementation{ Brak }
    \sholder{Zespół projektowy (UOB 01) }
    \reqrelated{Logowanie do systemu (WF 02)}
\end{requirementstab}

    \section{Interfejs z otoczeniem}

\begin{requirementstab}[label={tab:requirements:chat_gpt_3_5},caption={Wymagania na interfejs z otoczeniem dla integracji z ChatGPT 3.5}]
    \id{I01}
    \priority{S – should}
    \name{ChatGPT 3.5}
    \descr{Aplikacja zostaje połączona z API ChatGPT, aby użytkownik miał możliwość wygenerowania treści fiszki na podstawie podanego słowa.}
    \acceptcrit{Prawidłowo generuje definicje dla wysyłanych słów kluczowych.}
    \inputdata{Słowo podane przez użytkownika.}
    \preconditions{Aplikacja połączona z API ChatGPT.}
    \postconditions{ChatGPT zwraca definicję podanego słowa.}
    \exceptions{Brak połączenia z ChatGPT.}
    \implementation{Punkt 7.3.13}
    \sholder{Zespół projektowy - UOB 01 }
    \reqrelated{Generowanie treści fiszki na podstawie słów kluczowych (WF 16)}
\end{requirementstab}

    \section{Wymagania pozafunkcjonalne}

    \subsection{Instrukcja korzystania z trybu sterowania głosem}

\begin{requirementstab}[label={tab:requirements:voice_control_instruction},caption={Wymaganie pozafunkcjonalne dla instrukcji korzystania z trybu sterowania głosem}]
    \id{WN01}
    \priority{M – must}
    \name{Instrukcja korzystania z trybu sterowania głosem}
    \descr{Instrukcja zawiera komendy, wraz z opisem, których użytkownik korzysta w momencie uruchomienia trybu sterowania głosem.}
    \acceptcrit{Nowy użytkownik po zapoznaniu się z instrukcją, jest w stanie korzystać z trybu sterowania głosem.}
    \sholder{Zespół projektowy (UOB 01) }
    \reqrelated{Sterowanie talią przy użyciu mowy (WF 10)}
\end{requirementstab}

    \subsection{Limit błędów dotyczących logowania}

\begin{requirementstab}[label={tab:requirements:login_error_limit},caption={Wymaganie pozafunkcjonalne dla limitu błędów logowania}]
    \id{WN02}
    \priority{S – should}
    \name{Limit błędów dotyczących logowania}
    \descr{Po wpisaniu 3 razy niepoprawnych danych logowania w aplikacji pojawi się okienko informujące o blokadzie aplikacji na 1 minutę, okienko zawiera przycisk zmiany hasła. Przycisk przekierowuje do podstrony zawierającej pole do podania e-mail na które przyjdzie wiadomość dotycząca zmiany hasła.}
    \acceptcrit{Użytkownik otrzymuje wiadomość e-mail, która przekierowuje go do formularza zmiany hasła.}
    \sholder{Zespół projektowy (UOB 01) }
    \reqrelated{Resetowanie hasła (WF 017)}
\end{requirementstab}

    \subsection{Dostępność systemu}

\begin{requirementstab}[label={tab:requirements:system_availability},caption={Wymaganie pozafunkcjonalne dla dostępności systemu}]
    \id{WN03}
    \priority{M – must}
    \name{Dostępność systemu}
    \descr{System powinien być dostępny 7 dni w tygodniu, 24 godziny na dobę.}
    \acceptcrit{System działa przez 7 dni bez żadnych awarii. Co godzinę sprawdzane jest połączenie z systemem.}
    \sholder{Zespół projektowy (UOB 01) }
    \reqrelated{}
\end{requirementstab}

\subsection{Responsywność}

\begin{requirementstab}[label={tab:requirements:responsiveness},caption={Wymaganie pozafunkcjonalne dla responsywności systemu}]
    \id{WN04}
    \priority{M – must}
    \name{Responsywność}
    \descr{System musi być responsywnym, dostosowujący się do wielkości okienka lub urządzenia.}
    \acceptcrit{System w pełni dostosuje się do dostępnej wielkości okienka.}
    \sholder{Zespół projektowy (UOB 01) }
    \reqrelated{}
\end{requirementstab}

    \subsection{Kompatybilność}

\begin{requirementstab}[label={tab:requirements:compatibility},caption={Wymaganie pozafunkcjonalne dla kompatybilności z przeglądarkami internetowymi}]
    \id{WN05}
    \priority{M – must}
    \name{Kompatybilność}
    \descr{System musi być kompatybilny z różnymi przeglądarkami internetowymi.}
    \acceptcrit{System w pełni działa i ukazuje się w wybranej przeglądarce.}
    \sholder{Zespół projektowy (UOB 01) }
    \reqrelated{}
\end{requirementstab}

    \subsection{Hashowanie haseł}

\begin{requirementstab}[label={tab:requirements:password_hashing},caption={Wymaganie pozafunkcjonalne dla hashowania haseł}]
    \id{WN06}
    \priority{M – must}
    \name{Hashowanie haseł}
    \descr{Hasła użytkowników hashowane przy użyciu algorytmu sha256.}
    \acceptcrit{Hasło w bazie danych zostanie zaszyfrowane.}
    \sholder{Zespół projektowy (UOB 01) }
    \reqrelated{Rejestracja konta użytkownika (WF 01) \newline
Logowanie do systemu (WF 02)}
\end{requirementstab}

    \section{Wymagania na środowisko docelowe}

    \subsection{Przeglądarka}

\begin{requirementstab}[label={tab:requirements:browser},caption={Wymagania na środowisko docelowe dla przeglądarek internetowych}]
    \id{ŚD 01}
    \priority{M – must}
    \name{Przeglądarka}
    \descr{Chrome wersja 110.0.0.0, Firefox wersja 120.0, Microsoft Edge 115.0.0.0}
    \acceptcrit{Strona internetowa uruchamia się.}
    \sholder{Zespół projektowy (UOB 01) }
    \reqrelated{}
\end{requirementstab}

    \subsection{System Android 10 i iOS 16}

\begin{requirementstab}[label={tab:requirements:mobile_systems},caption={Wymagania na środowisko docelowe dla systemów mobilnych}]
    \id{ŚD 02}
    \priority{M – must}
    \name{System Android 10 i iOS 16}
    \descr{Aplikacja działa na urządzeniach mobilnych z systemem Android w wersji 10 i wyższych, w przypadku iOS w wersji 16 i wyższych.}
    \acceptcrit{Aplikacja uruchamia się na urządzeniach Android w wersji 10 i wyższych, w przypadku iOS w wersji 16 i wyższych.}
    \sholder{Zespół projektowy (UOB 01) }
    \reqrelated{}
\end{requirementstab}

    \subsection{Kontenery dockerowe}

\begin{requirementstab}[label={tab:requirements:docker_containers},caption={Wymagania na środowisko docelowe dla kontenerów dockerowych}]
    \id{ŚD 03}
    \priority{M – must}
    \name{Kontenery dockerowe}
    \descr{Dwa kontenery: jeden zawierający backend wraz z bazą danych, drugi zawierający aplikację webową.}
    \acceptcrit{Łączność pomiędzy kontenerami.}
    \sholder{Zespół projektowy (UOB 01) }
    \reqrelated{}
\end{requirementstab}

    \subsection{Baza danych}

\begin{requirementstab}[label={tab:requirements:database},caption={Wymagania na środowisko docelowe dla bazy danych}]
    \id{ŚD 04}
    \priority{M – must}
    \name{Baza danych}
    \descr{Mariadb - relacyjna baza danych w wersji 11.0. Będzie przechowywana w kontenerze razem z backend’em.}
    \acceptcrit{Prawidłowo tworzące się obiekty modeli, stabilna łączność z backendem.}
    \sholder{Zespół projektowy (UOB 01) }
    \reqrelated{}
\end{requirementstab}

    \subsection{Python}

\begin{requirementstab}[label={tab:requirements:python},caption={Wymagania na środowisko docelowe dla Pythona}]
    \id{ŚD 05}
    \priority{M – must}
    \name{Python}
    \descr{Python w wersji 3.10 będzie odpowiedzialny za poprawne działanie backendu.}
    \acceptcrit{Backend reaguje na otrzymywane requesty.}
    \sholder{Zespół projektowy (UOB 01) }
    \reqrelated{}
\end{requirementstab}

    \subsection{Node.js}

\begin{requirementstab}[label={tab:requirements:node},caption={Wymagania na środowisko docelowe dla Node.js}]
    \id{ŚD 06}
    \priority{M – must}
    \name{Node.js}
    \descr{Minimalna wersja Node.js dla aplikacji webowej oraz mobilnej to wersja 16. Będzie on odpowiedzialny za działania aplikacji webowej, jak i mobilnej.}
    \acceptcrit{Poprawne uruchomienie aplikacji webowej i mobilnej.}
    \sholder{Zespół projektowy (UOB 01) }
    \reqrelated{}
\end{requirementstab}

    \subsection{System operacyjny}

\begin{requirementstab}[label={tab:requirements:operating_system},caption={Wymagania na środowisko docelowe dla systemu operacyjnego}]
    \id{ŚD 07}
    \priority{M – must}
    \name{System operacyjny}
    \descr{System operacyjny Ubuntu w wersji 20, w którym zainicjowane będą kontenery dockerowe oraz środowisko z zapleczem technicznym projektu.}
    \acceptcrit{System stabilnie utrzymuje połączenie między użytkownikami a aplikacją, obsługuje środowisko techniczne projektu z zoptymalizowanym zużyciem zasobów oraz spełnia podstawowe wymagania zabezpieczeń bezpieczeństwa sieciowego.}
    \sholder{Zespół projektowy (UOB 01) }
    \reqrelated{ŚD 03 \newline ŚD 04 \newline ŚD 05 \newline ŚD 06}
\end{requirementstab}








