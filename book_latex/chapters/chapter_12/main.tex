\chapter{Podsumowanie}

Tworzenie projektu postawiło przed każdym członkiem zespołu nowe wyzwania programistyczne i grupowe. Pozwoliło nam to na poszerzenie naszych umiejętności programistycznych, analitycznych i twórczych. Różne podejście członków zespołu pozwoliło nam na poprawienie naszych umiejętności współpracy.

\section{Problemy i wyzwania}
Realizacja rozległego projektu informatycznego obejmującego równoczesną implementację działających ze sobą nawzajem systemów jest skomplikowanym przedsięwzięciem, które niesie ze sobą szereg problemów i wyzwań. Te w podjętym projekcie wynikały głównie ze specyfiki zastosowanych technologii oraz w dużej mierze z braku wcześniejszego doświadczenia członków zespołu z taką skalą projektu. W niniejszym rozdziale przybliżone zostaną najważniejsze problemy i wyzwania napotkane w trakcie implementacji.

\subsection{Pycharm Community Edition}
Pierwszym poważnym spośród napotkanych problemów były ograniczone możliwości PyCharm Community Edition, środowisko w wersji bezpłatne nie ma możliwości połączenia się z WSL (Windows Subsystem for Linux), na którym znajdowało się środowisko Ubuntu wraz z projektem do uruchomienia. Rozwiązaniem problem było skorzystanie z pełnej wersji PyCharm.

\subsection{Kontener bazy danych}
Następnym problemem było skonfigurowanie aplikacji na komputerze jednego z członków projektu. W momencie, w którym była dokonywana próba uruchomienia kontenera dockerowego dla bazy danych, pojawiał się błąd związany z brakiem połączenia z mysql. Po trzydniowej analizie problemu okazało się, że hasło do bazy danych w zmiennych środowiskowych było nieaktualne i różniło się od poprzedniego hasła jednym znakiem.

\subsection{Loader w aplikacji mobilnej}
Jednym z problemów który zahamował rozwój aplikacji mobilnej był błąd spowodowany przez Loader. Przyczyną było użycie biblioteki "moti", która po wywołaniu powodowała błąd o kodzie "useContext is null" uniemożliwiając kompilacje aplikacji dla systemu iOS. Możliwym rozwiązaniem naprawienia tego błędu jest użycie innej biblioteki.

\subsection{WSL i model nlp}
Model nlp nie działa na WSL. Po dodaniu do aplikacji modelu nlp okazało się, że aplikacja nie przestaje działać. Backend się zawiesza i zostaje zwrócona informacja segmentation fault, który oznacza brak dostępu do pamięci. Prawdopodobnym powodem wystąpienia problemu jest brak dostępu wsl do pełnych zasobów komputera. Jedynym rozwiązaniem było uruchomienie projektu na komputerze z zainstalowanym linuxem.

\subsection{Obsługa przy pomocy Mozilla Firefox}
Strona internetowa nie działa poprawnie w przeglądarce Mozilla Firefox. Z nieznanych przyczyn pozycjonowanie talli i użytkowników nie wczytuje się poprawnie i talie wychodzą poza ekran. Firefox ma ograniczone wsparcie dla Web Speech API,  co uniemożliwia korzystanie z funkcjonalności pozwalającej sterować talię fiszek używając komend głosowych.

\subsection{Api i HTTPS}
Przy realizacji rozproszonej infrastruktury wewnątrz serwera, trudna okazała się implementacja pełnego HTTPS. Strona korzystała w protokołu HTTPS, jednak łączyła się z api przez protokół HTTP. Przeglądarki internetowe mogą traktować taką zawartość jako “niebezpieczną”. Wymagane jest wtedy wyłączenie zabezpieczeń przeglądarki aby korzystanie ze strony było możliwe. Rozwiązanie tego problemu wymagało zaimplementowania zaawansowanej konfiguracji Nginx która pozwoliła przekierowywać cały ruch z domeny po https na api.

\subsection{Zatrzymanie nasłuchiwania mikrofonu}
Na stronie webowej po zatrzymaniu trybu sterowania głosem mikrofon jest jeszcze aktywny przez kilka sekund i strona reaguje na wypowiedziane przez użytkownika komendy głosowe. Próby zmiany zachowania działania mikrofonu tak, aby po zatrzymaniu kontroli głosowej przestał od razu nasłuchiwać nie przyniosły oczekiwanych rezultatów.

\subsection{Strumieniowanie głosu w aplikacji Expo Go}
W późnym etapie projektu napotkano problem związany ze strumieniowaniem głosu w aplikacji mobilnej. Przy pomocy platformy Expo Go nie jest to możliwe. Aby rozwiązać ten problem zdecydowano się na implementację obsługi głosowej za pomocą przetwarzania powtarzanych co kilka sekund nagrań. Implementacja rozwiązania ze strumieniem głosu bezpośrednio musiałoby wiązać się z gruntowną przebodową całego środowiska aplikacji mobilnej.

\section{Zmiany w trakcie realizacji}
Proces planowania i projektowania podjętego w projekcie systemu informatycznego wyprzedził o kilka miesięcy implementacje kluczowych funkcjonalności i założeń. Z powodu napotkanych wyzwań oraz nieprzewidzianych problemów doszło do konfrontacji rzeczywistości ze sprecyzowanymi wcześniej założeniami. Z tego powodu w projekcie musiało dojść do kilku opisanych w niniejszej sekcji zmian.

\subsection{Limit logowań}
Priorytet wymagania dotyczący limitu logowania został zmniejszony z must na should, ze względu na to, że aplikacja nie przechowuje żadnych danych wrażliwych użytkowników i zamiast tego wymagania został podniesiony priorytet generowania treści fiszek na podstawie słów kluczowych.

\subsection{Generowanie definicji przez ChatGTP}
Opis wymagania dotyczącego generowania treści przez ChatGPT został zmieniony. W pierwotnej wersji ta funkcjonalność miała działać tak, że użytkownik wpisywał w pole kilka słów i kluczowych i na tej podstawie ChatGPT miał zwrócić treść, ale takie podejście doprowadziłoby do generowania losowych treści. Zmieniona funkcjonalność działa tak, że użytkownik wpisuje w fiszek cokolwiek by chciał następnie aby wygenerować odpowiedź od ChatGPT klika przycisk generuj i treść wpisana w fiszkę jest wysyłana do chatu i ten na podstawie otrzymanej wiadomości zwraca odpowiedzi w postaci wygenerowanego tekstu odnoszącego się do zawartości fiszki.

\subsection{System operacyjny w infrastrukturze serwerowej}
W trakcie implementacji infrastruktury serwerowej i przenoszeniem projektu do chmury zmieniono wymagania dotyczące docelowego systemu operacyjnego. W fazie planowanie wybrano system CentOS 8. W późnej fazie projektu przy wdrażaniu wirtualnej maszyny na platformie chmurowej okazało się, że Azure proponuje jedynie starsze wersje CentOS 7 których wsparcie kończy się w czerwcu 2024. Z tego powodu zdecydowano się wybrać system Ubuntu 20.04 z możliwością podniesienia wersji do Ubuntu 22.04 ze znacznie dłuższym okresem wsparcia.


\section{Porzucone pomysły}
Faza planowania dla tak dużego projektu informatycznego wygenerowała wiele pomysłów i rozwiązań. Niektóre nie znalazły finalnego zastosowania. Te zostały opisane w niniejszej sekcji.

\subsection{Dyktowanie treści fiszki}
Zespół rozważał możliwość dodania funkcji dyktowanie treści fiszki przy użyciu mowy. Użytkownik po kliknięciu ikony mikrofonu miałby możliwość dyktowania treści fiszki. Zespół zrezygnował z tego pomysłu ze względu na problemy z poprawnym przekonwertowaniem mowy na tekst. Utworzony tekst nie zawsze odpowiadał temu, co mówi użytkownik.

\subsection{Sterowanie głosowe przez ChatGPT}
Przy pierwszym podejściu do sterowania talii przy użyciu komend głosowych został wykorzystany ChatGPT. Po przetworzeniu na tekstu słów wypowiedzianych przez użytkownika, treść była wysyłana do api. ChatGPT oprócz wypowiedzianych przez użytkownika słów otrzymywał również zestaw komend, na podstawie otrzymanych informacji miał za zadanie zwrócić komendę, która najbardziej pasuje semantycznie do wypowiedzianych przez użytkownika słów. Na podstawie otrzymanej informacji zwrotnej aplikacja wykonywała określoną akcję. Przy kilku pierwszych wysłanych requestach to podejście się sprawdzało, niestety jeżeli użytkownik w zbyt krótkim czasie wykonał za kilka komend głosowych, to request wysyłane do ChatGPT były blokowane i zwracany był bład statusu "error 500". To ograniczenie zmusiło nas do zrezygnowania z używania czatu do sterowania głosowego.

\subsection{Aktywacja konta i reset hasła}
W końcowej fazie projektu zrezygnowaliśmy również z aktywacji konta i resetu hasła. Funkcjonalności te zostały pozostawione na koniec i choć ich implementacja nie powinna stanowić wyzwania, częściowo były nawet zaimplementowane w projekcie,  uznaliśmy o ich odrzuceniu z uwagi na naglący czas i chęć skupienia na dopracowaniu istotniejszych funkcjonalności


\section{Przyszłość projektu}
Aktualnie nie przewiduje się zorganizowanej komercjalizacji projektu. Pozostanie on jednak dalej utrzymywany w oparciu o inne, bardziej korzystne rozwiązania chmurowe lub lokalną infrastrukture serwerową. Plan komercjalizacji i rozwoju systemu Fishki jest uzależniony od zdobycia bazy użytkowników oraz ich informacji zwrotnej. W przypadku zmiany decyzji, aplikacja zostanie rozbudowana o nowe funkcjonalności niezbędne do funkcjonowania w środowisku produkcyjnym. Zostanie dodana możliwość zresetowania hasła w przypadku gdy użytkownik zapomni hasła do logowania. Dodanie osiągnięć związanych z tworzeniem, udostępnianiem i ilością pobranych talii, jest to kolejna mechanika, która ma na celu zwiększenie aktywności użytkowników. Następnym rozpatrywanym ulepszeniem aplikacji jest dodanie narzędzi sztucznej inteligencji, które pozwolą na generowanie talii fiszek z dokumentów przesłanych przez użytkownika. Użytkownik będzie miał możliwość wrzucenia dokumentu w formacie pdf, csv lub xml i na podstawie zawartości dokumentu zostanie utworzona talia fiszek. Biorąc pod uwagę rozwiązania komercjalizacji w konkurencyjnych systemach, najlepszym rozważanym sposobem monetyzacji, który nie zniechęci użytkowników powinien działać w oparciu o dobrowolne darowizny.


\section{Zakończenie}
Niniejszy projekt udało się wypracować zgodnie z założonymi wymaganiami i wizją, która pozostała w dużym stopniu niezmienna. Podjęcie się wytworzenia systemu o tak wysokiej złożoności było dla nas wszystkich wyzwaniem z jakim nie mieliśmy do czynienia wcześniej. Pomimo wielu wyzwań oraz ograniczeń, udało się zrealizować Fishki, aplikację oferującą unikatowe funkcjonalności dla tego typu rozwiązania.
