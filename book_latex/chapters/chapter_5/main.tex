\newcolumntype{P}[1]{>{\centering\arraybackslash}p{#1}}

% creates point list for a member
\newcommand{\memberpointlist}[6]{
    \textbf{#1:}
    \begin{itemize}[leftmargin=17.5mm]
    \item #2;
    \item #3;
    \ifthenelse{{ \equal {#5} {} }}{\item #4.}{\item #4;}
    \ifthenelse{{ \equal {#5} {} }}{}{\ifthenelse{{ \equal {#6} {} }}{\item #5.}{\item #5;}}
    \ifthenelse{{ \equal {#6} {} }}{}{\item #6.}
    \end{itemize}
}

\newcommand{\gettitle}[1] {
    \centering \textbf{#1}
}

\newcommand{\gettableheaders} {
    \hline
    \centering \textbf{Jakub Żurawski} & \textbf{Daniel Klimowski} & \textbf{Oliwier Kossak} & \textbf{Wiktor Krieger} \\
}

\newcommand{\createtablerow}[4] {
    \hline
    \centering \ifthenelse{{ \equal {#1} {} }}{-}{#1} & \ifthenelse{{ \equal {#2} {} }}{-}{#2} & \ifthenelse{{ \equal {#3} {} }}{-}{#3} & \ifthenelse{{ \equal {#4} {} }}{-}{#4} \\
    \hline
}

\begin{chap5}
    \chapter{Organizacja pracy}


    \section{Harmonogram pracy}

    \gettitle{Październik 2023}
    \begin{longtable}{|P{2.3cm}|P{2.3cm}|P{2.3cm}|P{2.3cm}|}
        \gettableheaders
        \createtablerow{Omawianie z członkami zaspołu schematu budowy aplikacji np. jakich narzędzi użyć, czy gdzie monitorować postęp}{Omawianie z członkami zaspołu schematu budowy aplikacji np. jakich narzędzi użyć, czy gdzie monitorować postęp}{Omawianie z członkami zaspołu schematu budowy aplikacji np. jakich narzędzi użyć, czy gdzie monitorować postęp}{Omawianie z członkami zaspołu schematu budowy aplikacji np. jakich narzędzi użyć, czy gdzie monitorować postęp}
    \end{longtable}

    \gettitle{Listopad-Grudzień 2023}
    \begin{longtable}{|P{2.3cm}|P{2.3cm}|P{2.3cm}|P{2.3cm}|}
        \gettableheaders
        \createtablerow{Spotakania organizacyjne}{Spotakania organizacyjne}{Spotakania organizacyjne}{Spotakania organizacyjne}
        \createtablerow{Wypełnianie dokumentacji technicznej}{Wypełnianie dokumentacji technicznej i utworzenie diagram architektury}{Wypełnianie dokumentacji technicznej}{Wypełnianie dokumentacji technicznej}
        \createtablerow{Tworzenie diagramów}{Analiza kosztów i doboru odpowiedniego dostawcy usług chmurowych}{Tworzenie mocków aplikacji mobilnej}{Tworzenie mocków aplikacji aplikacji}
    \end{longtable}

    \gettitle{Styczeń 2024}
    \begin{longtable}{|P{2.3cm}|P{2.3cm}|P{2.3cm}|P{2.3cm}|}
        \gettableheaders
        \createtablerow{Spotakania organizacyjne}{Spotakania organizacyjne}{Spotakania organizacyjne}{Spotakania organizacyjne}
        \createtablerow{Poprawa i aktualizacja dokumentacji technicznej}{Poprawa i aktualizacja dokumentacji technicznej}{Poprawa i aktualizacja dokumentacji technicznej}{Poprawa i aktualizacja dokumentacji technicznej}
        \createtablerow{Poprawa diagramów}{Nauka Javascriptu i Reacta}{Tworzenie mocków aplikacji mobilnej}{Tworzenie mocków aplikacji aplikacji}
        \createtablerow{Utworzenie wstępnego uruchamiającego się api, opakowanie api w konterze dockerowym}{Konfiguracja środowiska iOS}{Dodanie modeli dla api i utworzenie connectora do bazy danych}{Przegląd artykułów naukowych}
    \end{longtable}

    \gettitle{Luty-Marzec 2024}
    \begin{longtable}{|P{2.3cm}|P{2.3cm}|P{2.3cm}|P{2.3cm}|}
        \gettableheaders
        \createtablerow{Spotkania organizacyjne}{Spotkania organizacyjne}{Spotkania organizacyjne}{Spotkania organizacyjne}
        \createtablerow{Implementacji komend terminalowych (cli) i poprawa dependecji RoleChecker'a}{Tworzenie ekranów aplikacji mobilnej (logowania, rejestracji i przypomnienia hasła)}{Naprawa środowiska FastAPI}{Konfiguracja środowiska}
        \createtablerow{Poprawa bezpieczeństwa api, poprawa middleware'ów, dodanie loggera i endpointów autoryzujących}{Konspekt pracy dyplomowej}{Poprawa działania token autoryzującego i blacklist tokens}{Utworzenie okno logowania (webówka)}
        \createtablerow{Stawianie i konfiguracja środowiska Android}{Utworzenie strony domowej dla aplikacji mobilnej}{Dodanie endpointów dla flash cards}{Projektowanie ikon}
        \createtablerow{Naprawa aplikacji mobilnej, poprawa doboru wersji potrzebnych paczek}{Piszanie rodziału drugiego 'Omówienie problemu'}{Poprawki w modelach bazy danych, dodanie crudów do endpointu deck}{Poprawa mocków aplikacji webowej}
        \createtablerow{Implementacja metody rquest z interfejsem, i dodanie paneli użytkownika}{Czyszczenie kodu, przerzucenie regex'ów do kalalogi validator}{Utworzenie kontenera dla aplikacji webowej}{Utworzenie tła aplikacji webowej}
    \end{longtable}

    \gettitle{Kwiecień-Maj 2024}
    \begin{longtable}{|P{2.3cm}|P{2.3cm}|P{2.3cm}|P{2.3cm}|}
        \gettableheaders
        \createtablerow{Pisanie rodziału 5}{Pisanie rodziałów 2, 3, 4}{Pisanie rodziałów 3, 4, 5, 10 pracy}{Tworzenie ikon}
        \createtablerow{Utworzenie widoku dla aktualizacji danych użytkownika}{Utworzenie widoku my decks}{Utworzenie strony tworzenia fiszek, utworzenie dla nich endpointów i połączenia jej całości z api}{Utworzenie widoku profilu użytkownika dla aplikacji webowej}
        \createtablerow{To be continue...}{To be continue...}{To be continue...}{To be continue...}
    \end{longtable}


    \begin{flushleft}

        \section{Wybrana metodyka}

        \par Nasza wybrana metodyka nazywa się \textbf{modelem przyrostowo-ewolucyjnym}.
        Jest to połączeniem dwóch modeli, przyrostowego i ewolucyjnego.
        W naszym projekcie implementacja kodu odbywałą się w trakcie spintu, a każdy kolejny sprint był planowany po
        zakończeniu aktualnie prowadzonego (przerwa między sprintami), razem z wypełnianiem dokumentacji.
        Dzięki temu projekt był rozwijany i ewoulował cały czas, nie tylko w sprintach ale, i w trakcie przerw.
        Zalety takiego rozwiązania to przedewszystkim łatwe podejmowanie działań nad rozwojem aplikacji,
        wczesną i stopniową dystrybucja oprogramowania, stały wgląd nad wyglądem oraz funkcjonowaniem aplikacji,
        czy poprawy naniesione w trakcie produckji w zwiazku ze zmianami wymagań.


        \section{Zespół i podział obowiązków}

        \memberpointlist{Daniel Klimowski}{Budowanie aplikacji mobilnej}{Konfiguracja Azure}{Wypełnianie dokumentacji technicznej}{}{}

        \memberpointlist{Oliwier Kossak}{Budowa aplikacji webowej}{Budowa aplikacji backendowej}{Tworzenie modeli sql}{Wypełnianie dokumentacji technicznej}{Mockupy aplikacji mobilnej}

        \memberpointlist{Wiktor Krieger}{Budowa aplikacji webowej}{Mockupy aplikacji webowej i mobilnej}{Wypełnianie dokumentacji technicznej}{Tworzenie ikon dla aplikacji}{}

        \memberpointlist{Jakub Żurawski}{Budowa aplikacji mobilnej}{Budowa aplikacji backendowej}{Wypełnianie dokumentacji technicznej}{Tworzenie diagramów}{Tworzenie modeli sql}
    \end{flushleft}
\end{chap5}