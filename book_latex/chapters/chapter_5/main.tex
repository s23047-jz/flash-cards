\newcolumntype{P}[1]{>{\centering\arraybackslash}p{#1}}

% creates headers for a member
\newcommand{\member}[6]{
    \textbf{#1} \\
    \hline #2 \\
    \hline #3 \\
    \hline #4
    \ifthenelse{{ \equal {#5} {} }}{}{\\ \hline #5}
    \ifthenelse{{ \equal {#6} {} }}{}{\\ \hline #6}
}

% creates tables for a member
% param #1 -> member
\newcommand{\membertable}[1]{
    \begin{tabularx}{1\textwidth}{|>{\centering\arraybackslash}X|}
        \hline
        #1
        \\
        \hline
    \end{tabularx}
    \vspace{2em}
}

\newcommand{\gettableheaders}[1] {
    \hline
    \centering \textbf{#1} & \textbf{Jakub Żurawski} & \textbf{Daniel Klimowski} & \textbf{Oliwier Kossak} & \textbf{Wiktor Krieger} \\
}

\newcommand{\createtablerow}[5] {
    \hline
    \centering #1 & \ifthenelse{{ \equal {#2} {} }}{brak}{#2} & \ifthenelse{{ \equal {#3} {} }}{brak}{#3} & \ifthenelse{{ \equal {#4} {} }}{brak}{#4} & \ifthenelse{{ \equal {#5} {} }}{brak}{#5} \\
    \hline
}

\begin{chap5}
    \chapter{Organizacja pracy}

    \section{Harmonogram pracy}

    \begin{longtable}{|P{2.3cm}|P{2.3cm}|P{2.3cm}|P{2.3cm}|P{2.3cm}|}
        \gettableheaders{Październik 2023}
        \createtablerow{20-21}{Dyskusja nad wyborem projektu}{Dyskusja nad wyborem projektu}{Dyskusja nad wyborem projektu}{Dyskusja nad wyborem projektu}
        \createtablerow{24}{Spotkanie organizacyjne}{Spotkanie organizacyjne}{Spotkanie organizacyjne}{Spotkanie organizacyjne}
        \createtablerow{25}{Uzupełnienie DZW}{Uzupełnienie DZW}{Uzupełnienie DZW}{Uzupełnienie DZW}
        \createtablerow{26-28}{Nanoszenie poprawek do DZW oraz SWS}{}{Nanoszenie poprawek do DZW}{}
        \createtablerow{31}{}{}{}{Wstępna konfiguracja projektu w jirze}
    \end{longtable}

    \begin{longtable}{|P{2.3cm}|P{2.3cm}|P{2.3cm}|P{2.3cm}|P{2.3cm}|}
        \gettableheaders{Listopad 2023}
        \createtablerow{08}{Spotkanie organizacyjne}{Spotkanie organizacyjne}{Spotkanie organizacyjne}{Spotkanie organizacyjne i projektowanie warstwy wizualnej aplikacji mobilnej w Canva}
        \createtablerow{12}{}{}{Utworzenie wstępnego diagramu użycia przypadków}{}
        \createtablerow{13}{}{}{Uzupełnienie podpunktów 4 i 5 w DZW}{}
        \createtablerow{14}{}{Uzupełnianie dokumentu SWS}{Uzupełnienie dokumentu SWS}{Projektowanie logo}
        \createtablerow{16}{}{}{Uzupełnianie dokumentu SWS}{}
        \createtablerow{19}{}{Rozwinięcie punktów 3, 5, 8 w DZW}{}{}
        \createtablerow{20}{Tworzenie diagramu przypadków użycia}{Nanoszenie poprawek do DZW}{}{}
        \createtablerow{21}{Poprawa diagramu przypadków użycia i utworzenie diagramy aktywności}{}{Uzupełnienie dokumentu SWS}{}
        \createtablerow{22}{Poprawa diagramu przypadków użycia i spotkanie organizacyjne}{}{Uzupełnienie dokumentu SWS}{}
        \createtablerow{27}{}{}{Rozwinięcie podpunktu 3 w DZW, analiza konkurencji}{}
        \createtablerow{29}{Spotkanie organizacyjne}{Spotkanie organizacyjne}{Analiza działania aplikacji mobilnej i spotkanie organizacyjne}{Analiza kokurencji i spotkanie organizacyjne}
    \end{longtable}

    \begin{longtable}{|P{2.3cm}|P{2.3cm}|P{2.3cm}|P{2.3cm}|P{2.3cm}|}
        \gettableheaders{Grudzień 2023}
        \createtablerow{6}{}{}{Tworzenie i poprawa mockow w canvie}{Tworzenie i poprawa mockow w canvie}
        \createtablerow{9}{Spotkanie organizacyjne, omawiania poprawy diagramów}{}{Tworzenie i poprawa mockow w canvie, spotkanie organizacyjne}{}
        \createtablerow{12}{Przerzucenie diagramów do innego programu oraz poprawia diagramów}{}{Tworzenie i poprawa mockow w canvie}{}
        \createtablerow{13}{}{}{Tworzenie i poprawa mockow w canvie}{}
        \createtablerow{19}{}{}{Tworzenie i poprawa mockow w canvie}{}
        \createtablerow{20}{}{}{Tworzenie i poprawa mockow w canvie}{}
        \createtablerow{28}{}{}{Tworzenie i poprawa mockow w canvie}{}
        \createtablerow{29}{}{}{Poprawa SWS}{}
        \createtablerow{31}{}{}{Poprawa SWS}{}
    \end{longtable}

    \begin{longtable}{|P{2.3cm}|P{2.3cm}|P{2.3cm}|P{2.3cm}|P{2.3cm}|}
        \gettableheaders{Styczeń 2024}
        \createtablerow{1}{}{}{Tworzenie diagramu ERD}{}
        \createtablerow{3}{Spotkanie organizacyjne i poprawa diagramu użycia przypadków}{Spotkanie organizacyjne}{Spotkanie organizacyjne}{Spotkanie organizacyjne}
        \createtablerow{4}{}{}{Spotkanie organizacyjne i poprawa, i tworzenie mocków w canvie}{Spotkanie organizacyjne}
        \createtablerow{5}{Utworzenie wstępnego diagramu modeli i relacji dla bazy danych i poprawa SWS}{}{Poprawa i tworzenie mocków w canvie}{}
        \createtablerow{6}{Poprawa SWS}{Poprawa SWS}{Poprawa SWS}{}
        \createtablerow{8}{}{}{}{Projektowanie warstwy wizualnej aplikacji mobilnej w Canva}
        \createtablerow{9}{Poprawa SWS}{}{Poprawa SWS}{Projektowanie warstwy wizualnej aplikacji mobilnej w Canva}
        \createtablerow{10}{}{}{}{Projektowanie warstwy wizualnej aplikacji mobilnej w Canva}
        \createtablerow{14}{Tworzenie środowiska FastAPI}{Poprawa DZW i SWS oraz pisanie karty projektu}{Poprawa DZW i tworzenie środowiska FastAPI}{}
        \createtablerow{17}{Aktualizacja ReadMe i poprawa silnika do łączenia z baza danych, plus utworzenie zmiennych środowiskowych}{}{Utworzenie silnika do łączności z bazą danych}{}
        \createtablerow{23}{Poprawa diagramu użycia przypadkow i SWS}{Spotkanie organizacyjne}{Spotkanie organizacyjne}{Spotkanie organizacyjne i przegląd artykółów naukowych}
        \createtablerow{25}{}{}{}{Aktualizacja dokumentu z Artykułami}
        \createtablerow{30}{}{}{Spotkanie organizacyjne}{}
        \createtablerow{31}{Poprawa plików dockerowych}{Spotkanie organizacyjne}{Spotkanie organizacyjne}{}
    \end{longtable}

    \begin{longtable}{|P{2.3cm}|P{2.3cm}|P{2.3cm}|P{2.3cm}|P{2.3cm}|}
        \gettableheaders{Luty 2024}
        \createtablerow{13}{Implementcja cli(click scripts)}{}{Nauka HTML i Css, i spotkanie orgaznizacyjne}{}
        \createtablerow{14}{}{}{Nauka HTML, Css i Javascriptu}{}
        \createtablerow{15}{}{}{Nauka HTML, Css i Javascriptu}{}
        \createtablerow{16}{}{}{Nauka Javascriptu i poprawa środowiska FastAPI}{}
        \createtablerow{18}{Poprawa cli, endpointów autoryzacyjnych oraz działania customowych middleware'ów}{}{Poprawa importów dla api}{}
        \createtablerow{21}{Spotkanie organizacyjne}{Spotkanie orgaznizacyjne}{Nauka Javascriptu i spotkanie organizacyjne}{Spotkanie organizacyjne}
        \createtablerow{22}{}{}{Nauka Reacta}{}
        \createtablerow{23}{}{}{Nauka Reacta}{}
        \createtablerow{24}{}{}{Nauka Reacta i naprawa importów api}{}
        \createtablerow{25}{}{}{Nauka Reacta}{}
        \createtablerow{27}{Dodanie loggera oraz poprawa ścieżek api}{}{Nauka Reacta}{}
        \createtablerow{28}{Aktualizacja dependencies i endpointów dla api}{}{Nauka Reacta}{}
    \end{longtable}

    \begin{longtable}{|P{2.3cm}|P{2.3cm}|P{2.3cm}|P{2.3cm}|P{2.3cm}|}
        \gettableheaders{Marzec 2024}
        \createtablerow{1}{Pomoc w uruchomieniu środowiska do pracy u Oliwiera}{}{Naprawa środowiska FastAPI}{}
        \createtablerow{2}{}{}{Naprawa środowiska FastAPI}{}
        \createtablerow{3}{}{Tworzenie ekrana logowania dla aplikacji mobilnej FC-23}{Naprawa środowiska FastAPI i naprawa błędu związanego z tokenem autoryzującym}{}
        \createtablerow{4}{Stawianie środowiska android}{Tworzenie ekrana logowania dla aplikacji mobilnej FC-23}{}{}
        \createtablerow{5}{Usuwanie błędów związanych z nieprawidłowym package'em dla mobilki}{Tworzenie ekrana logowania, rejestracji i przypomnienia hasła dla aplikacji mobilnej FC-23}{Naprawa błedu z tokenem i blacklist tokens}{}
        \createtablerow{6}{Poprawa dependencies oraz dodanie nowch endpointów dla api (FC-28)}{}{Poprawki modeli, dodanie crudów do endpointu decks}{}
        \createtablerow{7}{}{}{Rozwiązanie konfliktów na githubie}{}
        \createtablerow{9}{Utworzenie wstępnych fixtur (FC-30), dodanie flake8 (FC-31), dodanie serwisu do logowania i rejestracji (FC-32)}{}{}{}
        \createtablerow{10}{}{}{Dodanie endpointów dla flash cards}{}
        \createtablerow{11}{}{Spotkanie organizacyjne}{Spotkanie organizacyjne i poprawa modeli bazy danych FC-36 chyba}{Spotkanie organizacyjne}
        \createtablerow{12}{}{FC-41, FC-42, FC-39}{FC-22 tworzenie kontenera node js}{}
        \createtablerow{14}{}{}{}{}
        \createtablerow{15}{}{}{FC-40 i pomoc w postawieniu projektu u Wiktora}{Konfiguracja projektu i środowiska}
        \createtablerow{16}{}{}{FC-40}{}
        \createtablerow{17}{}{}{FC-40 i uruchamianie środowiska React u Wiktora}{Konfiguracja środowiska React'a, FC-20 i projekt tła strony web}
        \createtablerow{18}{}{}{Spotkanie organizacyjne}{}
        \createtablerow{19}{}{COMMITY???????????????}{FC-40}{}
        \createtablerow{20}{}{Konfiguracja latex'a i utworzenie wstępnej karty projektu, FC-39}{FC-40 i FC-47}{Projektowanie ikon}
        \createtablerow{21}{}{}{FC-47}{}
        \createtablerow{22}{}{}{FC-47}{}
        \createtablerow{23}{}{Konfiguracja api i dockera na maszynie lokalnej}{}{}
        \createtablerow{24}{}{FC-39}{}{}
        \createtablerow{25}{}{}{}{FC-20 (integracja z backendem)}
        \createtablerow{26}{}{Konspekt pracy dyplomowej, FC-39 i spotkanie organizacyjne}{Spotkanie organizacyjne i FC-47}{}
        \createtablerow{27}{}{}{FC-47}{}
        \createtablerow{28}{}{}{FC-47}{}
        \createtablerow{29}{}{}{FC-47}{}
        \createtablerow{30}{}{}{FC-47}{}
    \end{longtable}

    \section{Wybrana metodyka}

    \par Nasza wybrana metodyka nazywa się \textbf{modelem przyrostowo-ewolucyjnym}.
    Jest to połączeniem dwóch modeli, przyrostowego i ewolucyjnego.
    W naszym projekcie implementacja kodu odbywałą się w trakcie spintu, a każdy kolejny sprint był planowany po
    zakończeniu aktualnie prowadzonego (przerwa między sprintami), razem z wypełnianiem dokumentacji.
    Dzięki temu projekt był rozwijany i ewoulował cały czas, nie tylko w sprintach ale, i w trakcie przerw.
    Zalety takiego rozwiązania to przedewszystkim łatwe podejmowanie działań nad rozwojem aplikacji,
    wczesną i stopniową dystrybucja oprogramowania, stały wgląd nad wyglądem oraz funkcjonowaniem aplikacji,
    czy poprawy naniesione w trakcie produckji w zwiazku ze zmianami wymagań.

    \section{Zespół i podział obowiązków}
    \centering
    \membertable{\member{Daniel Klimowski}{Budowanie aplikacji mobilnej}{Konfiguracja Azure}{Wypełnianie dokumentacji technicznej}{}{}}

    \membertable{\member{Oliwier Kossak}{Budowa aplikacji webowej}{Budowa aplikacji backendowej}{Tworzenie modeli sql}{Wypełnianie dokumentacji technicznej}{Mockupy aplikacji mobilnej}}

    \membertable{\member{Wiktor Krieger}{Budowa aplikacji webowej}{Mockupy aplikacji webowej i mobilnej}{Wypełnianie dokumentacji technicznej}{Tworzenie ikon dla aplikacji}{}}

    \membertable{\member{Jakub Żurawski}{Budowa aplikacji mobilnej}{Budowa aplikacji backendowej}{Wypełnianie dokumentacji technicznej}{Tworzenie diagramów}{Tworzenie modeli sql}}
\end{chap5}