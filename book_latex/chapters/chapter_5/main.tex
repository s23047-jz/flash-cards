\usepackage{xifthen}

\newcommand{\member}[6]{
    \textbf{#1} \\
    \hline #2 \\
    \hline #3 \\
    \hline #4
    \ifthenelse{{ \equal {#5} {} }}{}{\\ \hline #5}
    \ifthenelse{{ \equal {#6} {} }}{}{\\ \hline #6}
}

\newcommand{\membertable}[1]{
    \begin{tabularx}{1\textwidth}{|>{\centering\arraybackslash}X|}
        \hline
        #1
        \\
        \hline
    \end{tabularx}
    \vspace{2em}
}

\begin{chap5}
    \chapter{Organizacja pracy}

    \section{Wybrana metodyka}
    \par \textbf{Agile} - główną zaletą tej metodyki jest elastyczność, która bardzo ułatwia podejmowanie działań,
    czy sterowanie zakresem pracy.

    \par Implementacja kodu odbyła się metodyką Scrum opartą na zasadach Agile. Umożliwiało łatwe podejmowanie działań i sterowaniem
    rozwoju aplikacji poprzez planowanie celu na koniec każdego sprintu. Natomiast wypełnianie dokumentacji odbywało się w przerwach pomiędzy sprintami.

    \section{Zespół i podział obowiązków}
    \centering
    \membertable{\member{Daniel Klimowski}{Budowanie aplikacji mobilnej}{Konfiguracja Azure}{Wypełnianie dokumentacji technicznej}{}{}}

    \membertable{\member{Oliwier Kossak}{Budowa aplikacji webowej}{Budowa aplikacji backendowej}{Tworzenie modeli sql}{Wypełnianie dokumentacji technicznej}{Mockupy aplikacji mobilnej}}

    \membertable{\member{Wiktor Krieger}{Budowa aplikacji webowej}{Mockupy aplikacji webowej i mobilnej}{Wypełnianie dokumentacji technicznej}{Tworzenie ikon dla aplikacji}{}}

    \membertable{\member{Jakub Żurawski}{Budowa aplikacji mobilnej}{Budowa aplikacji backendowej}{Wypełnianie dokumentacji technicznej}{Tworzenie diagramów}{Tworzenie modeli sql}}
\end{chap5}