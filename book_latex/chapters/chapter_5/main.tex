\newcolumntype{P}[1]{>{\centering\arraybackslash}p{#1}}

% creates point list for a member
\newcommand{\memberpointlist}[6]{
    \textbf{#1:}
    \begin{itemize}[leftmargin=17.5mm]
    \item #2;
    \item #3;
    \ifthenelse{{ \equal {#5} {} }}{\item #4.}{\item #4;}
    \ifthenelse{{ \equal {#5} {} }}{}{\ifthenelse{{ \equal {#6} {} }}{\item #5.}{\item #5;}}
    \ifthenelse{{ \equal {#6} {} }}{}{\item #6.}
    \end{itemize}
}

\newcommand{\gettitle}[1] {
    \centering \textbf{#1}
}

\newcommand{\gettableheaders} {
    \hline
    \centering \textbf{Jakub Żurawski} & \textbf{Daniel Klimowski} & \textbf{Oliwier Kossak} & \textbf{Wiktor Krieger} \\
    \hline
}

\newcommand{\createtablerow}[4] {
    \hline
    \centering \ifthenelse{{ \equal {#1} {} }}{-}{#1} & \ifthenelse{{ \equal {#2} {} }}{-}{#2} & \ifthenelse{{ \equal {#3} {} }}{-}{#3} & \ifthenelse{{ \equal {#4} {} }}{-}{#4} \\
    \hline
}

\begin{chap5}
    \chapter{Organizacja pracy}


    \section{Harmonogram pracy}


    Nie mieliśmy ściśle określonego harmonogramu pracy.
    W początkowej fazie budowy projektu skupiliśmy się na części planowania. Najpierw ustaliliśmy aby zebrać odpowiednie
    informacje, wypełnić dokumentację i stworzyć założenia. W drugiej fazie przeszliśmy do implementacji,
    razem z pisaniem książki, a postęp produkcyjny i zarządzanie śledziliśmy za pomocą Jiry.

    \gettitle{Październik 2023 - Styczeń 2024}
    \begin{longtable}{|P{2.3cm}|P{2.3cm}|P{2.3cm}|P{2.3cm}|}
        \rowcolor{lightgray}\multicolumn{4}{|c|}{\textbf{Planowanie}} \\
        \gettableheaders
        \createtablerow{Analiza wymagań}{Analiza wymagań}{Analiza wymagań}{Analiza wymagań}
        \createtablerow{Dobór odpowiednich technologii i narzędzi pracy}{Dobór odpowiednich technologii i narzędzi pracy}{Dobór odpowiednich technologii i narzędzi pracy}{Dobór odpowiednich technologii i narzędzi pracy}
        \createtablerow{Wypełnienie dokumentacji techniczne}{Wypełnienie dokumentacji techniczne}{Wypełnienie dokumentacji techniczne}{Wypełnienie dokumentacji techniczne}
        \createtablerow{Określenie celów i zakresu projektu}{Określenie celów i zakresu projektu}{Określenie celów i zakresu projektu}{Określenie celów i zakresu projektu}
        \createtablerow{Utworzenie diagramów}{}{Utworzenie mocków aplikacji}{Utworzenie mocków i ikon aplikacji}
    \end{longtable}

    \gettitle{Luty 2024 - Czerwien 2024}
    \begin{longtable}{|P{2.3cm}|P{2.3cm}|P{2.3cm}|P{2.3cm}|}
        \rowcolor{lightgray}\multicolumn{4}{|c|}{\textbf{Implementacja}} \\
        \gettableheaders
        \createtablerow{Implementacja backendu}{Implementacja aplikacji mobilnej}{Implementacja backendu}{Implementacja aplikacji webowej}
        \createtablerow{Implementacja aplikacji mobilnej}{Konfiguracja serwera}{Implementacja aplikacji webowej}{Pisanie książki}
        \createtablerow{Pisanie książki}{Pisanie książki}{Pisanie książki}{}
    \end{longtable}

\end{chap5}

\section{Wybrana metodyka}

\indent
Zastosowaną w projekcie metodyką jest \textbf{model przyrostowo-ewolucyjny}.
Jest to połączeniem dwóch modeli, przyrostowego i ewolucyjnego. Tworzenie projektu zostało podzielone na dwa etapy:
fazie planowania i fazie implementacji. Faza planowania polegała na zbieraniu wymagań, wypełnianiu dokumentacji i
tworzeniu mocukapów aplikacji. Faza ta podbyła się w podejściu ewolucyjnym, co pozowliło na adapatcję wymagań,
podczas tworzenia wspólnej wizji projektu.\cite{roger2004} Faza implementacji polegała na tworzeniu i
pisaniu aplikacji. Faza ta odbywała się w trakcie spintów, a każdy kolejny sprint był planowany po
zakończeniu aktualnie prowadzonego (przerwa między sprintami).
Zalety takiego rozwiązania to przede wszystkim łatwe podejmowanie działań nad rozwojem aplikacji,
wczesną i stopniową dystrybucja oprogramowania, stały wgląd nad wyglądem oraz funkcjonowaniem aplikacji, czy
możliwość nanoszenia poprawek w trakcie produckji w zwiazku ze zmianami wymagań.

\section{Zespół i podział obowiązków}

\memberpointlist{Daniel Klimowski}{Budowanie aplikacji mobilnej}{Konfiguracja Azure}{Wypełnianie dokumentacji technicznej}{}{}

\memberpointlist{Oliwier Kossak}{Budowa aplikacji webowej}{Budowa aplikacji backendowej}{Tworzenie modeli sql}{Wypełnianie dokumentacji technicznej}{Mockupy aplikacji mobilnej}

\memberpointlist{Wiktor Krieger}{Budowa aplikacji webowej}{Mockupy aplikacji webowej i mobilnej}{Wypełnianie dokumentacji technicznej}{Tworzenie ikon dla aplikacji}{}

\memberpointlist{Jakub Żurawski}{Budowa aplikacji mobilnej}{Budowa aplikacji backendowej}{Wypełnianie dokumentacji technicznej}{Tworzenie diagramów}{Tworzenie modeli sql}
