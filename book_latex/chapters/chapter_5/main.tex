\newcolumntype{P}[1]{>{\centering\arraybackslash}p{#1}}

% creates headers for a member
\newcommand{\member}[6]{
    \textbf{#1} \\
    \hline #2 \\
    \hline #3 \\
    \hline #4
    \ifthenelse{{ \equal {#5} {} }}{}{\\ \hline #5}
    \ifthenelse{{ \equal {#6} {} }}{}{\\ \hline #6}
}

% creates tables for a member
% param #1 -> member
\newcommand{\membertable}[1]{
    \begin{tabularx}{1\textwidth}{|>{\centering\arraybackslash}X|}
        \hline
        #1
        \\
        \hline
    \end{tabularx}
    \vspace{2em}
}

\newcommand{\gettableheaders}[1] {
    \hline
    \centering \textbf{#1} & \textbf{Jakub Żurawski} & \textbf{Daniel Klimowski} & \textbf{Oliwier Kossak} & \textbf{Wiktor Krieger} \\
}

\newcommand{\createtablerow}[5] {
    \hline
    \centering #1 & \ifthenelse{{ \equal {#2} {} }}{brak}{#2} & \ifthenelse{{ \equal {#3} {} }}{brak}{#3} & \ifthenelse{{ \equal {#4} {} }}{brak}{#4} & \ifthenelse{{ \equal {#5} {} }}{brak}{#5} \\
    \hline
}

\begin{chap5}
    \chapter{Organizacja pracy}

    \section{Harmonogram pracy}

    \begin{longtable}{|P{2.3cm}|P{2.3cm}|P{2.3cm}|P{2.3cm}|P{2.3cm}|}
        \gettableheaders{Październik 2023}
        \createtablerow{20-21}{Dyskusja nad wyborem projektu}{Dyskusja nad wyborem projektu}{Dyskusja nad wyborem projektu}{Dyskusja nad wyborem projektu}
        \createtablerow{24}{Spotkanie organizacyjne}{Spotkanie organizacyjne}{Spotkanie organizacyjne}{Spotkanie organizacyjne}
        \createtablerow{25}{Uzupełnienie DZW}{Uzupełnienie DZW}{Uzupełnienie DZW}{Uzupełnienie DZW}
        \createtablerow{26-28}{Nanoszenie poprawek do DZW oraz SWS}{}{Nanoszenie poprawek do DZW}{}
        \createtablerow{31}{}{}{}{Wstępna konfiguracja projektu w jirze}
    \end{longtable}

    \begin{longtable}{|P{2.3cm}|P{2.3cm}|P{2.3cm}|P{2.3cm}|P{2.3cm}|}
        \gettableheaders{Listopad 2023}
        \createtablerow{08}{Spotkanie organizacyjne}{Spotkanie organizacyjne}{Spotkanie organizacyjne}{Spotkanie organizacyjne i projektowanie warstwy wizualnej aplikacji mobilnej w Canva}
        \createtablerow{12}{}{}{Utworzenie wstępnego diagramu użycia przypadków}{}
        \createtablerow{13}{}{}{Uzupełnienie podpunktów 4 i 5 w DZW}{}
        \createtablerow{14}{}{Uzupełnianie dokumentu SWS}{Uzupełnienie dokumentu SWS}{Projektowanie logo}
        \createtablerow{16}{}{}{Uzupełnianie dokumentu SWS}{}
        \createtablerow{19}{}{Rozwinięcie punktów 3, 5, 8 w DZW}{}{}
        \createtablerow{20}{Tworzenie diagramu przypadków użycia}{Nanoszenie poprawek do DZW}{}{}
        \createtablerow{21}{Poprawa diagramu przypadków użycia i utworzenie diagramy aktywności}{}{Uzupełnienie dokumentu SWS}{}
        \createtablerow{22}{Poprawa diagramu przypadków użycia i spotkanie organizacyjne}{}{Uzupełnienie dokumentu SWS}{}
        \createtablerow{27}{}{}{Rozwinięcie podpunktu 3 w DZW, analiza konkurencji}{}
        \createtablerow{29}{Spotkanie organizacyjne}{Spotkanie organizacyjne}{Analiza działania aplikacji mobilnej i spotkanie organizacyjne}{Analiza kokurencji i spotkanie organizacyjne}
    \end{longtable}

    \begin{longtable}{|P{2.3cm}|P{2.3cm}|P{2.3cm}|P{2.3cm}|P{2.3cm}|}
        \gettableheaders{Grudzień 2023}
        \createtablerow{6}{}{}{Tworzenie i poprawa mockow w canvie}{Tworzenie i poprawa mockow w canvie}
        \createtablerow{9}{Spotkanie organizacyjne, omawiania poprawy diagramów}{}{Tworzenie i poprawa mockow w canvie, spotkanie organizacyjne}{}
        \createtablerow{12}{Przerzucenie diagramów do innego programu oraz poprawia diagramów}{}{Tworzenie i poprawa mockow w canvie}{}
        \createtablerow{13}{}{}{Tworzenie i poprawa mockow w canvie}{}
        \createtablerow{19}{}{}{Tworzenie i poprawa mockow w canvie}{}
        \createtablerow{20}{}{}{Tworzenie i poprawa mockow w canvie}{}
        \createtablerow{28}{}{}{Tworzenie i poprawa mockow w canvie}{}
        \createtablerow{29}{}{}{Poprawa SWS}{}
        \createtablerow{31}{}{}{Poprawa SWS}{}
    \end{longtable}

    \section{Wybrana metodyka}

    \par Nasza wybrana metodyka nazywa się \textbf{modelem przyrostowo-ewolucyjnym}.
    Jest to połączeniem dwóch modeli, przyrostowego i ewolucyjnego.
    W naszym projekcie implementacja kodu odbywałą się w trakcie spintu, a każdy kolejny sprint był planowany po
    zakończeniu aktualnie prowadzonego (przerwa między sprintami), razem z wypełnianiem dokumentacji.
    Dzięki temu projekt był rozwijany i ewoulował cały czas, nie tylko w sprintach ale, i w trakcie przerw.
    Zalety takiego rozwiązania to przedewszystkim łatwe podejmowanie działań nad rozwojem aplikacji,
    wczesną i stopniową dystrybucja oprogramowania, stały wgląd nad wyglądem oraz funkcjonowaniem aplikacji,
    czy poprawy naniesione w trakcie produckji w zwiazku ze zmianami wymagań.

    \section{Zespół i podział obowiązków}
    \centering
    \membertable{\member{Daniel Klimowski}{Budowanie aplikacji mobilnej}{Konfiguracja Azure}{Wypełnianie dokumentacji technicznej}{}{}}

    \membertable{\member{Oliwier Kossak}{Budowa aplikacji webowej}{Budowa aplikacji backendowej}{Tworzenie modeli sql}{Wypełnianie dokumentacji technicznej}{Mockupy aplikacji mobilnej}}

    \membertable{\member{Wiktor Krieger}{Budowa aplikacji webowej}{Mockupy aplikacji webowej i mobilnej}{Wypełnianie dokumentacji technicznej}{Tworzenie ikon dla aplikacji}{}}

    \membertable{\member{Jakub Żurawski}{Budowa aplikacji mobilnej}{Budowa aplikacji backendowej}{Wypełnianie dokumentacji technicznej}{Tworzenie diagramów}{Tworzenie modeli sql}}
\end{chap5}