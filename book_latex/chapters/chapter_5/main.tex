\newcolumntype{P}[1]{>{\centering\arraybackslash}p{#1}}

% creates point list for a member
\newcommand{\memberpointlist}[6]{
    \textbf{#1:}
    \begin{itemize}[leftmargin=17.5mm]
        \item #2;
        \item #3;
        \ifthenelse{{ \equal {#5} {} }}{\item #4.}{\item #4;}
        \ifthenelse{{ \equal {#5} {} }}{}{\ifthenelse{{ \equal {#6} {} }}{\item #5.}{\item #5;}}
        \ifthenelse{{ \equal {#6} {} }}{}{\item #6.}
    \end{itemize}
}

\newcommand{\gettableheaders}[1] {
    \hline
    \centering \textbf{#1} & \textbf{Jakub Żurawski} & \textbf{Daniel Klimowski} & \textbf{Oliwier Kossak} & \textbf{Wiktor Krieger} \\
}

\newcommand{\createtablerow}[5] {
    \hline
    \centering #1 & \ifthenelse{{ \equal {#2} {} }}{brak}{#2} & \ifthenelse{{ \equal {#3} {} }}{brak}{#3} & \ifthenelse{{ \equal {#4} {} }}{brak}{#4} & \ifthenelse{{ \equal {#5} {} }}{brak}{#5} \\
    \hline
}

\begin{chap5}
    \chapter{Organizacja pracy}

    \section{Harmonogram pracy}

    \begin{longtable}{|P{2.3cm}|P{2.3cm}|P{2.3cm}|P{2.3cm}|P{2.3cm}|}
        \gettableheaders{Październik 2023}
        \createtablerow{20-21}{Dyskusja nad wyborem projektu}{Dyskusja nad wyborem projektu}{Dyskusja nad wyborem projektu}{Dyskusja nad wyborem projektu}
        \createtablerow{24}{Spotkanie organizacyjne}{Spotkanie organizacyjne}{Spotkanie organizacyjne}{Spotkanie organizacyjne}
        \createtablerow{25}{Uzupełnienie DZW}{Uzupełnienie DZW}{Uzupełnienie DZW}{Uzupełnienie DZW}
        \createtablerow{26-28}{Nanoszenie poprawek do DZW oraz SWS}{}{Nanoszenie poprawek do DZW}{}
        \createtablerow{31}{}{}{}{Wstępna konfiguracja projektu w jirze}
    \end{longtable}

    \section{Wybrana metodyka}

    \par Nasza wybrana metodyka nazywa się \textbf{modelem przyrostowo-ewolucyjnym}.
    Jest to połączeniem dwóch modeli, przyrostowego i ewolucyjnego.
    W naszym projekcie implementacja kodu odbywałą się w trakcie spintu, a każdy kolejny sprint był planowany po
    zakończeniu aktualnie prowadzonego (przerwa między sprintami), razem z wypełnianiem dokumentacji.
    Dzięki temu projekt był rozwijany i ewoulował cały czas, nie tylko w sprintach ale, i w trakcie przerw.
    Zalety takiego rozwiązania to przedewszystkim łatwe podejmowanie działań nad rozwojem aplikacji,
    wczesną i stopniową dystrybucja oprogramowania, stały wgląd nad wyglądem oraz funkcjonowaniem aplikacji,
    czy poprawy naniesione w trakcie produckji w zwiazku ze zmianami wymagań.

    \section{Zespół i podział obowiązków}

    \memberpointlist{Daniel Klimowski}{Budowanie aplikacji mobilnej}{Konfiguracja Azure}{Wypełnianie dokumentacji technicznej}{}{}

    \memberpointlist{Oliwier Kossak}{Budowa aplikacji webowej}{Budowa aplikacji backendowej}{Tworzenie modeli sql}{Wypełnianie dokumentacji technicznej}{Mockupy aplikacji mobilnej}

    \memberpointlist{Wiktor Krieger}{Budowa aplikacji webowej}{Mockupy aplikacji webowej i mobilnej}{Wypełnianie dokumentacji technicznej}{Tworzenie ikon dla aplikacji}{}

    \memberpointlist{Jakub Żurawski}{Budowa aplikacji mobilnej}{Budowa aplikacji backendowej}{Wypełnianie dokumentacji technicznej}{Tworzenie diagramów}{Tworzenie modeli sql}
\end{chap5}