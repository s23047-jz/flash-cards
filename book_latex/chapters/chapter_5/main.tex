\newcolumntype{P}[1]{>{\centering\arraybackslash}p{#1}}

% creates point list for a member
\newcommand{\memberpointlist}[6]{
    \textbf{#1:}
    \begin{itemize}[leftmargin=17.5mm]
    \item #2;
    \item #3;
    \ifthenelse{{ \equal {#5} {} }}{\item #4.}{\item #4;}
    \ifthenelse{{ \equal {#5} {} }}{}{\ifthenelse{{ \equal {#6} {} }}{\item #5.}{\item #5;}}
    \ifthenelse{{ \equal {#6} {} }}{}{\item #6.}
    \end{itemize}
}

\newcommand{\gettitle}[1] {
    \centering \textbf{#1}
}

\newcommand{\gettableheaders} {
    \hline
    \centering \textbf{Jakub Żurawski} & \textbf{Daniel Klimowski} & \textbf{Oliwier Kossak} & \textbf{Wiktor Krieger} \\
    \hline
}

\newcommand{\createtablerow}[4] {
    \hline
    \centering \ifthenelse{{ \equal {#1} {} }}{-}{#1} & \ifthenelse{{ \equal {#2} {} }}{-}{#2} & \ifthenelse{{ \equal {#3} {} }}{-}{#3} & \ifthenelse{{ \equal {#4} {} }}{-}{#4} \\
    \hline
}

\begin{chap5}
    \chapter{Organizacja pracy}


    \section{Harmonogram pracy}

    Harmonogram pracy nie został określony na wstępie projektu. Na początku prac, koncentracja skupiła się głównie na etapie planowania. początkowym priorytetem było wypełnienie dokumentacji projektowej, zebranie informacji oraz zdefiniowanie założeń i wymagań. Druga faza projektu poświęcona była implementacji przy równoczesnym redagowaniu ksiązki dyplomowej. Postępy w produkcji oraz zarządzanie projektem monitorowane były za pomocą systemu Jira.

    \gettitle{Październik 2023 - Styczeń 2024}
    \begin{longtable}{|P{2.3cm}|P{2.3cm}|P{2.3cm}|P{2.3cm}|}
        \rowcolor{lightgray}\multicolumn{4}{|c|}{\textbf{Planowanie}} \\
        \gettableheaders
        \createtablerow{Analiza wymagań}{Analiza wymagań}{Analiza wymagań}{Analiza wymagań}
        \createtablerow{Dobór odpowiednich technologii i narzędzi pracy}{Dobór odpowiednich technologii i narzędzi pracy}{Dobór odpowiednich technologii i narzędzi pracy}{Dobór odpowiednich technologii i narzędzi pracy}
        \createtablerow{Wypełnienie dokumentacji technicznej}{Wypełnienie dokumentacji technicznej}{Wypełnienie dokumentacji technicznej}{Wypełnienie dokumentacji technicznej}
        \createtablerow{Określenie celów i zakresu projektu}{Określenie celów i zakresu projektu}{Określenie celów i zakresu projektu}{Określenie celów i zakresu projektu}
        \createtablerow{Utworzenie diagramów}{}{Utworzenie mockupów aplikacji}{Utworzenie mockupów i ikon aplikacji}
                \caption{Harmonogram planowania.}
    \end{longtable}

    \gettitle{Luty 2024 - Czerwien 2024}
    \begin{longtable}{|P{2.3cm}|P{2.3cm}|P{2.3cm}|P{2.3cm}|}
        \rowcolor{lightgray}\multicolumn{4}{|c|}{\textbf{Implementacja}} \\
        \gettableheaders
        \createtablerow{Implementacja backendu}{Implementacja aplikacji mobilnej}{Implementacja backendu}{Implementacja aplikacji webowej}
        \createtablerow{Implementacja aplikacji mobilnej}{Konfiguracja serwera}{Implementacja aplikacji webowej}{Pisanie książki}
        \createtablerow{Pisanie książki}{Pisanie książki}{Pisanie książki}{}
        \caption{Harmonogram implementacji.}
    \end{longtable}

\end{chap5}

\section{Wybrana metodyka}

\indent
Zastosowaną w projekcie metodyką jest \textbf{model przyrostowo-ewolucyjny}. Stanowi on połączenie dwóch modeli: przyrostowego oraz ewolucyjnego. Proces tworzenia projektu został podzielony na fazę planowania oraz fazę implementacji.
Faza planowania obejmowała zbieranie wymagań, wypełnianie dokumentacji oraz tworzenie mockupów aplikacji. Zrealizowano ją w podejściu ewolucyjnym, co pozwoliło na adaptację wymagań podczas tworzenia wspólnej wizji projektu \cite{roger2004}.
Faza implementacji koncentrowała się na technicznej realizacji projektu, tj. na tworzeniu aplikacji. Odbywała się w trakcie sprintów, z których każdy był planowany po zakończeniu poprzedniego.
Zalety takiego podejścia to przede wszystkim łatwość podejmowania działań nad rozwojem aplikacji, wczesna i stopniowa dystrybucja oprogramowania, stały wgląd w wygląd oraz funkcjonowanie aplikacji, a także możliwość nanoszenia poprawek w trakcie produkcji w związku ze zmianami wymagań.

\section{Zespół i podział obowiązków}

\memberpointlist{Daniel Klimowski}{Budowanie aplikacji mobilnej}{Konfiguracja Azure}{Wypełnianie dokumentacji technicznej}{}{}

\memberpointlist{Oliwier Kossak}{Budowa aplikacji webowej}{Budowa aplikacji backendowej}{Tworzenie modeli sql}{Wypełnianie dokumentacji technicznej}{Mockupy aplikacji mobilnej}

\memberpointlist{Wiktor Krieger}{Budowa aplikacji webowej}{Mockupy aplikacji webowej i mobilnej}{Wypełnianie dokumentacji technicznej}{Tworzenie ikon dla aplikacji}{}

\memberpointlist{Jakub Żurawski}{Budowa aplikacji mobilnej}{Budowa aplikacji backendowej}{Wypełnianie dokumentacji technicznej}{Tworzenie diagramów}{Tworzenie modeli sql}

\section{Analiza ryzyka}

Podrozdział przedstawia analizę ryzyka, która ma na celu określenie i zrozumienie potencjalnych zagrożeń, które mogą wpłynąć na proces tworzenia projektu.

\setstretch{1.0}

\begin{longtable}{|p{2.3cm}|p{2.3cm}|p{2.3cm}|p{2.3cm}|p{2.3cm}|p{2.3cm}|}
    \hline
\textbf{Zidentyfiko- wane ryzyko [20]} & \textbf{Symptomy} & \textbf{Środki / Działania zapobiegawcze i szacowany poziom trudności ich wdrożenia} & \textbf{Środki / Działania minimalizujące wpływ na projekt – już po jego wystąpieniu i szacowany poziom trudności ich wdrożenia (1-10)} & \textbf{Ranga ryzyka (im niższa, tym mniejszy negatywny wpływ na projekt)} & \textbf{Prawdopodo- bieństwo wystąpienia (1-100\%)} \\
\hline
Błędy w kodzie [O] & Błędy w kompilacji, wyniki testów nie pokrywają się z oczekiwanym rezultatem  & Szybka identyfikacja błędów, wykonywanie testów oprogramowania & Poprawa kodu, testowanie na bieżąco nowo implementowanych funkcjonalności (4) & 10 & 80\% \\
\hline
Awaria sprzętu deweloperskiego [S] & Sprzęt przestaje działać, brak możliwości odzyskania danych & Częste commity na repozytorium & Działanie na ostatniej wersji z repozytorium (1) & 7 & 10\% \\
\hline
Niedobór umiejętności w zespole [L] & Problemy jednostek w poszczególnych zadaniach & Uzupełnianie wiedzy & Pomoc innych członków zespołu (2) & 7 & 70\% \\
\hline
Niezgodność czasowa zespołu [C]  & Osoba blokuje postęp nad projektem poprzez odpowiedzialność nad kluczowym elementem & Wcześniejsze zaplanowanie pracy & Wspólna praca nad kluczowym elementem zajęcie się zadaniami niepowiązanymi (5) & 6 & 80\% \\
\hline
Wybór nieodpowiedniej technologii [T] & Planowane rozwiązania nie są osiągalne przez wykorzystywane technologie & Dogłębna analiza potrzebnych technologii i ich możliwości/ograniczeń & Dopasowanie alternatywnych możliwych rozwiązań (8) & 5 & 50\% \\
\hline
Niedoszacowa- nie budżetu potrzebnego do utrzymania infrastruktury [B] & Budżet zbliża się do wyczerpania przed zaplanowanym terminem  & Analiza cenników wykorzystywanych produktów & Próba pozyskania inwestorów (10) & 10 & 80\% \\
\hline
ChatGPT 3.5 generuje bezsensowną treść dotycząca zagadnienia o którą został zapytany przez użytkownika [F] & Generowane treści nie nawiązują do zagadnienia, o które ChatGPT został zapytany & Próba sprecyzowania zapytania, które zostało przesłane do ChatGPT w celu wygenerowania konkretniejszej treści & Użytkownik może zaakceptować lub odrzucić wygenerowaną treść. (4) & 7 & 50\% \\
\hline
    Przeglądarka niepoprawnie wczytuje stronę internetową [Ś] & Style strony nie wyglądają tak samo jak na stronie uruchomionej lokalnie lub na innych przeglądarkach. Pozycje elementów na stronie są inaczej rozmieszczone. & Próba dostosowania styli lub funkcjonalności do konkretnej przeglądarki.  & Uruchomienie strony internetowej na innej przeglądarce (3).   & 10 & 70\% \\
    \hline
    \caption{Analiza ryzyka.}
\end{longtable}

\setstretch{1.5}
