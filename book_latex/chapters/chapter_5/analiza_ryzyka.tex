\usepackage{ragged2e}

\section{Analiza ryzyka}


Podrozdział przedstawia analizę ryzyka, która ma na celu określenie i zrozumienie potencjalnych zagrożeń, które mogą wpłynąć na proces tworzenia projektu.

\setstretch{1.0}
\newpage
\begin{longtable}{|>{\raggedright\arraybackslash}p{2.4cm}|>{\raggedright\arraybackslash}p{2.4cm}|>{\raggedright\arraybackslash}p{2.4cm}|>{\raggedright\arraybackslash}p{2.4cm}|>{\raggedright\arraybackslash}p{2.4cm}|>{\raggedright\arraybackslash}p{2.4cm}|}
    \hline
    \textbf{Zidentyfiko -wane ryzyko [20]} & \textbf{Symptomy} & \textbf{Środki / Działania zapobiegawcze i szacowany poziom trudności ich wdrożenia} & \textbf{Środki / Działania minimalizujące wpływ na projekt – już po jego wystąpieniu i szacowany poziom trudności ich wdrożenia (1-10)} & \textbf{Ranga ryzyka (im niższa, tym mniejszy negatywny wpływ na projekt)} & \textbf{Prawdopodo -bieństwo wystąpienia (1-100\%)} \\
    \hline
    Błędy w kodzie [O] & Błędy w kompilacji, wyniki testów nie pokrywają się z oczekiwanym rezultatem  & Szybka identyfikacja błędów, wykonywanie testów oprogramowania & Poprawa kodu, testowanie na bieżąco nowo implementowanych funkcjonalności (4) & 10 & 80\% \\
    \hline
    Awaria sprzętu deweloperskiego [S] & Sprzęt przestaje działać, brak możliwości odzyskania danych & Częste commity na repozytorium & Działanie na ostatniej wersji z repozytorium (1) & 7 & 10\% \\
    \hline
    Niedobór umiejętności w zespole [L] & Problemy jednostek w poszczególnych zadaniach & Uzupełnianie wiedzy & Pomoc innych członków zespołu (2) & 7 & 70\% \\
    \hline
    Niezgodność czasowa zespołu [C]  & Osoba blokuje postęp nad projektem poprzez odpowiedzialność nad kluczowym elementem & Wcześniejsze zaplanowanie pracy & Wspólna praca nad kluczowym elementem zajęcie się zadaniami niepowiązanymi (5) & 6 & 80\% \\
    \hline
    Wybór nieodpowiedniej technologii [T] & Planowane rozwiązania nie są osiągalne przez wykorzystywane technologie & Dogłębna analiza potrzebnych technologii i ich możliwości lub ograniczeń & Dopasowanie alternatywnych możliwych rozwiązań (8) & 5 & 50\% \\
    \hline
    Niedoszacowanie budżetu potrzebnego do utrzymania infrastruktury [B] & Budżet zbliża się do wyczerpania przed zaplanowanym terminem  & Analiza cenników wykorzy -stywanych produktów & Próba pozyskania inwestorów (10) & 10 & 80\% \\
    \hline
    ChatGPT 3.5 generuje bezsensowną treść dotycząca zagadnienia o którą został zapytany przez użytkownika [F] & Generowane treści nie nawiązują do zagadnienia, o które ChatGPT został zapytany & Próba sprecyzowania zapytania, które zostało przesłane do ChatGPT w celu wygenerowania konkretniejszej treści & Użytkownik może zaakceptować lub odrzucić wygenerowaną treść. (4) & 7 & 50\% \\
    \hline
    Przeglądarka niepoprawnie wczytuje stronę internetową [Ś] & Style strony nie wyglądają tak samo jak na stronie uruchomionej lokalnie lub na innych przeglądarkach. Pozycje elementów na stronie są inaczej rozmieszczone. & Próba dostosowania styli lub funkcjonalności do konkretnej przeglądarki.  & Uruchomienie strony internetowej na innej przeglądarce (3).   & 10 & 70\% \\
    \hline
    \caption{Analiza ryzyka.}
\end{longtable}


\setstretch{1.5}
