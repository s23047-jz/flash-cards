\documentclass{sprz}
\usepackage[backend=biber,style=numeric-comp,sorting=none]{biblatex}
%\usepackage[backend=biber, style=numeric-comp, autocite=footnote]{biblatex}

\usepackage{enumitem}
\usepackage[T1]{fontenc}
\usepackage{setspace}
\usepackage{longtable}
\usepackage{float}
\usepackage{indentfirst}
\usepackage[all]{nowidow}
\usepackage{polyglossia}
\setmainlanguage[spacedguillemets=true]{polish}

\addbibresource{bibliography.bib}


%\setmainfont{Times New Roman}

\studfield{Informatyka}
\studtype{Zaoczne}
\title{Fishki - multiplatformowa aplikacja służąca do nauki pamięciowej}
\engtitle{Fishki - a multiplatform application for memory learning}
\acronym{Fishki}
\titledate{11.10.2023}
\supervisor{dr Puźniakowski Tadeusz}
\author{Kossak Oliwier}{s22018}{Sztuczna Inteligencja}{Niestacjonarny}
\author{Klimowski Daniel}{s18504}{Sztuczna Inteligencja}{Niestacjonarny}
\author{Krieger Wiktor}{s23638}{Sztuczna Inteligencja}{Niestacjonarny}
\author{Żurawski Jakub}{s23047}{Sztuczna Inteligencja}{Niestacjonarny}
\consultant{--- brak ---} % Koniecznie trzeba podać brak, albo wpisać konsultantów tak jak przy autorach
\projectgoals{Dostarczenie systemu do efektywnej nauki z wykorzystaniem techniki zapamiętywania w postaci fiszek. Cel projektu odpowiada na problem rozumiany jako trudności w organizacji oraz korzystania z materiałów służących do opanowywania pojęć. }
\productsandservices{Strona internetowa, aplikacja mobilna, serwer obsługujący utrzymanie i żywotność produktu }
\mainfunctionalities{Nauka metodą fishek, obsługa talii tj. zestawów fiszek, obsługa talii fiszek za pomocą mowy, tworzenie fiszek z pomocą sztucznej inteligencji. }
\successmeasure{Wytworzenie działającej strony internetowej i aplikacji mobilnej, opracowanie techniczne projektu z wykorzystaniem infrastruktury serwerowej, zaimplementowanie silnika sztucznej inteligencji, zrealizowanie wymagań systemowych na poziomie ‘wymagane’.}
\projlimitations{Zdalny charakter pracy nad projektem, budżet, brak doświadczenia w pracy nad projektem o zadanej złożoności, nauka i wykorzystanie nowych technologii.}
\date{\today}
\nabstract{
    TBC
}



\begin{document}

    \maketitle

    \makeprojectcard
    \makedeclaration

    \tableofcontents

    \setstretch{1.5}

    \chapter{Analiza wymagań}

\section{Wymagania ogólne}
\section{Wymagania funkcjonalne}
\section{Wymagania pozafunkcjonalne}
\section{Wymagania jakościowe i pozostałe}


    \chapter{Analiza wymagań}

\section{Wymagania ogólne}
\section{Wymagania funkcjonalne}
\section{Wymagania pozafunkcjonalne}
\section{Wymagania jakościowe i pozostałe}


    \chapter{Analiza wymagań}

\section{Wymagania ogólne}
\section{Wymagania funkcjonalne}
\section{Wymagania pozafunkcjonalne}
\section{Wymagania jakościowe i pozostałe}


  %  \chapter{Analiza wymagań}

\section{Wymagania ogólne}
\section{Wymagania funkcjonalne}
\section{Wymagania pozafunkcjonalne}
\section{Wymagania jakościowe i pozostałe}
 TRESC PRZENIESIONA DO ROZDZIALU 3

    \chapter{Analiza wymagań}

\section{Wymagania ogólne}
\section{Wymagania funkcjonalne}
\section{Wymagania pozafunkcjonalne}
\section{Wymagania jakościowe i pozostałe}


    \chapter{Analiza wymagań}

\section{Wymagania ogólne}
\section{Wymagania funkcjonalne}
\section{Wymagania pozafunkcjonalne}
\section{Wymagania jakościowe i pozostałe}


    \chapter{Analiza wymagań}

\section{Wymagania ogólne}
\section{Wymagania funkcjonalne}
\section{Wymagania pozafunkcjonalne}
\section{Wymagania jakościowe i pozostałe}


    \chapter{Analiza wymagań}

\section{Wymagania ogólne}
\section{Wymagania funkcjonalne}
\section{Wymagania pozafunkcjonalne}
\section{Wymagania jakościowe i pozostałe}


    \chapter{Analiza wymagań}

\section{Wymagania ogólne}
\section{Wymagania funkcjonalne}
\section{Wymagania pozafunkcjonalne}
\section{Wymagania jakościowe i pozostałe}


    \printbibliography[title={Bibliografia}]
%    \printbibliography[title={Bibliografia}, heading=bibintoc]

\end{document}
